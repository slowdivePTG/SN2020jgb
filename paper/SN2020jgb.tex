%% Beginning of file 'SN\,2020jgb.tex'
%% using aastex version 6.31
\documentclass[twocolumn]{aastex631}
\usepackage{CJK}
%\usepackage{lineno}
%\linenumbers

\newcommand{\sn}{SN\,2020jgb}
\newcommand{\trmax}{$t_{r_\mathrm{ZTF},\mathrm{max}}$}
\newcommand{\tfl}{$t_\mathrm{fl}$}
\newcommand{\Mch}{$M_\mathrm{Ch}$}
\newcommand{\kms}{$\mathrm{km}\,\mathrm{s}^{-1}$}
\newcommand{\Ni}{$^{56}\mathrm{Ni}$}
\newcommand{\Msun}{\mathrm{M_\odot}}
\newcommand{\adam}[1]{\textcolor{red}{[AAM: #1]}}
\newcommand{\chang}[1]{\textcolor{blue}{[Chang: #1]}}

\shorttitle{\sn}
\shortauthors{Liu et al.}
\graphicspath{{./}{figures/}}

\begin{document}
%\begin{CJK*}{UTF8}{gbsn}

\title{\sn: A Peculiar Type Ia Supernova Triggered by a Massive Helium-Shell Detonation in a Star-Forming Galaxy}

%main contributors
\author[0000-0002-7866-4531]{Chang~Liu}
\affiliation{Center for Interdisciplinary Exploration and Research in Astrophysics (CIERA), Department of Physics and Astronomy, Northwestern University, 1800 Sherman Road, Evanston, IL 60201, USA}

\author[0000-0001-9515-478X]{Adam~A.~Miller}
\affil{Center for Interdisciplinary Exploration and Research in Astrophysics (CIERA), Department of Physics and Astronomy, Northwestern University, 1800 Sherman Road, Evanston, IL 60201, USA}

\author[0000-0002-2028-9329]{Anya~E.~Nugent}
\affil{Center for Interdisciplinary Exploration and Research in Astrophysics (CIERA), Department of Physics and Astronomy, Northwestern University, 1800 Sherman Road, Evanston, IL 60201, USA}

\author[0000-0002-2028-9329]{Abigail~Polin}
\affil{The Observatories of the Carnegie Institution for Science, 813 Santa Barbara Street, Pasadena, CA 91101, USA}
\affil{TAPIR, Walter Burke Institute for Theoretical Physics, 350-17, Caltech, Pasadena, CA 91125, USA}

\author[0000-0002-8989-0542]{Kishalay~De}
\altaffiliation{NASA Einstein Fellow}
\affil{MIT-Kavli Institute for Astrophysics and Space Research, 77 Massachusetts Ave., Cambridge, MA 02139, USA}

\author[0000-0002-3389-0586]{Peter~E.~Nugent}
\affil{Department of Astronomy, University of California, Berkeley, CA 94720, USA}
\affil{Lawrence Berkeley National Laboratory, 1 Cyclotron Road, Berkeley, CA, 94720, USA}

\author[0000-0001-6797-1889]{Steve~Schulze}
\affil{The Oskar Klein Centre, Department of Physics, Stockholm University, Albanova University Center, SE 106 91 Stockholm, Sweden}

\author[0000-0002-3653-5598]{Avishay~Gal-Yam}
\affil{Department of particle physics and astrophysics, Weizmann Institute of Science, 76100 Rehovot, Israel}

\author[0000-0002-4223-103X]{Christoffer~Fremling}
\affiliation{Caltech Optical Observatories, California Institute of Technology, Pasadena, CA 91125, USA}
\affiliation{Division of Physics, Mathematics, and Astronomy, California Institute of Technology, Pasadena, CA 91125, USA}

%spectra
\author[0000-0003-3768-7515]{Shreya~Anand}
\affil{Cahill Center for Astrophysics, California Institute of Technology, Pasadena CA 91125, USA}

\author[0000-0002-8977-1498]{Igor~Andreoni}
\altaffiliation{Neil Gehrels Fellow}
\affil{Joint Space-Science Institute, University of Maryland, College Park, MD 20742, USA.}
\affil{Department of Astronomy, University of Maryland, College Park, MD 20742, USA.}
\affil{Astrophysics Science Division, NASA Goddard Space Flight Center, Mail Code 661, Greenbelt, MD 20771, USA}

\author[0000-0003-0526-2248]{Peter~Blanchard}
\affil{Center for Interdisciplinary Exploration and Research in Astrophysics (CIERA), Department of Physics and Astronomy, Northwestern University, 1800 Sherman Road, Evanston, IL 60201, USA}

\author[0000-0001-5955-2502]{Thomas~G.~Brink}
\affil{Department of Astronomy, University of California, Berkeley, CA 94720-3411, USA}

\author[0000-0002-2376-6979]{Suhail~Dhawan}
\affil{Institute of Astronomy and Kavli Institute for Cosmology, University of Cambridge, Madingley Road, Cambridge CB3 0HA, UK}

\author[0000-0003-3460-0103]{Alexei~V.~Filippenko}
\affil{Department of Astronomy, University of California, Berkeley, CA 94720-3411, USA}

\author[0000-0002-9770-3508]{Kate~Maguire}
\affil{School of Physics, Trinity College Dublin, The University of Dublin, Dublin 2, Ireland}

\author[0000-0001-8948-3456]{Tassilo~Schweyer}
\affil{The Oskar Klein Centre, Department of Astronomy, Stockholm University, AlbaNova, SE-106 91 Stockholm, Sweden}

\author[0000-0001-8023-4912]{Huei~Sears}
\affil{Center for Interdisciplinary Exploration and Research in Astrophysics (CIERA), Department of Physics and Astronomy, Northwestern University, 1800 Sherman Road, Evanston, IL 60201, USA}

\author[0000-0003-4531-1745]{Yashvi~Sharma}
\affil{Division of Physics, Mathematics, and Astronomy, California Institute of Technology, Pasadena, CA 91125, USA}

%builders
\author[0000-0002-3168-0139]{Matthew~J.~Graham}
\affiliation{Division of Physics, Mathematics, and Astronomy, California Institute of Technology, Pasadena, CA 91125, USA}

\author[0000-0001-5668-3507]{Steven~L.~Groom}
\affiliation{IPAC, California Institute of Technology, 1200 E. California Blvd, Pasadena, CA 91125, USA}

\author{David~Hale}
\affiliation{Caltech Optical Observatories, California Institute of Technology, Pasadena, CA 91125, USA}

\author[0000-0002-5619-4938]{Mansi M. Kasliwal}
\affil{Division of Physics, Mathematics, and Astronomy, California Institute of Technology, Pasadena, CA 91125, USA}

\author[0000-0001-5390-8563]{Shrinivas~R.~Kulkarni}
\affiliation{Division of Physics, Mathematics, and Astronomy, California Institute of Technology, Pasadena, CA 91125, USA}

\author[0000-0002-8532-9395]{Frank~J.~Masci}
\affiliation{IPAC, California Institute of Technology, 1200 E. California Blvd, Pasadena, CA 91125, USA}

\author[0000-0003-1227-3738]{Josiah~Purdum}
\affiliation{Caltech Optical Observatories, California Institute of Technology, Pasadena, CA 91125, USA}

\author[0000-0001-8861-3052]{Benjamin~Racine}
\affiliation{Aix Marseille Univ, CNRS/IN2P3, CPPM, Marseille, France}

\author[0000-0003-1546-6615]{Jesper~Sollerman}
\affiliation{The Oskar Klein Centre, Department of Astronomy, Stockholm University, AlbaNova, SE-106 91 Stockholm, Sweden}

\begin{abstract} 
%
The detonation of a thin ($\lesssim$0.03\,$\Msun$) helium shell (He-shell) atop a $\sim$$1\,\Msun$ white dwarf (WD) is a promising mechanism to explain normal Type Ia supernovae (SNe\,Ia), while thicker He-shells and less massive WDs may explain some recently observed peculiar SNe\,Ia. We present observations of \sn, a peculiar SN\,Ia discovered by the Zwicky Transient Facility (ZTF). Near maximum light, \sn\ is subluminous (ZTF $g$-band absolute magnitude $M_g\approx -18.2$\,mag) and shows an unusually red color ($g_\mathrm{ZTF}-r_\mathrm{ZTF}\approx 0.4$\,mag) due to the strong line-blanketing blueward of $\sim$5000\,\AA. These properties resemble those of SN\,2018byg, a peculiar SN\,Ia consistent with a ``thick He-shell'' double detonation (DDet) SN. Using detailed radiative transfer models, we show that the optical spectroscopic and photometric evolution of \sn\ are broadly consistent with a He-shell of $\sim$0.08\,$\Msun$ detonating above a carbon-oxygen WD of $\sim$0.87\,$\Msun$. Depending on the actual reddening \sn\ suffers in the host galaxy, the total progenitor mass (shell and core) could be up to $\sim$1.00\,$\Msun$. We detect a prominent absorption feature at $\sim$1\,\micron\ in the near-infrared (NIR) spectrum of \sn, which could originate from the unburnt helium in the outermost ejecta. While the sample size is limited, similar 1\,\micron\ features have been detected in all the thick He-shell DDet candidates with NIR spectra obtained to date. \sn\ is also the first subluminous thick He-shell DDet SN discovered in a star-forming galaxy, indisputably showing that He-shell DDet objects occur in both star-forming and passive galaxies, consistent with the normal SN\,Ia population.
%
\end{abstract}

\keywords{Supernovae (1668), Type Ia supernovae (1728), White dwarf stars (1799), Observational astronomy (1145), Surveys (1671)}

\section{Introduction} \label{sec:intro}
It has been clear for decades that Type Ia supernovae (SNe\,Ia) are caused by the thermonuclear explosions of carbon-oxygen (C/O) white dwarfs (WDs) in binary systems \citep[see][for a review]{Maoz_2014}. Nevertheless, the nature of the binary companion, as well as how it ignites the WD, remains highly uncertain. 

The helium-shell (He-shell) double detonation (DDet) scenario is one of the most promising channels to produce SNe\,Ia. In this scenario, the WD accretes from a companion to develop a helium-rich shell, which, after becoming sufficiently massive, could detonate. Such a detonation sends a shock wave into the C/O core to trigger a runaway thermonuclear explosion that inevitably disrupts and destroys the entire WD \citep{Nomoto_1982a, Nomoto_1982b, Woosley_1986, Livne_1990, Woosley_1994, Livne_1995}. This DDet mechanism can produce explosions of WDs below the Chandrasekhar-mass (\Mch).

There are several observational benchmarks for He-shell DDet SNe. Shortly after the ignition of the He-shell, the decay of radioactive material in the helium ashes may power a detectable flash \citep{Woosley_1994,Fink_DD_2010,Kromer_DD_2010}. The Fe-group elements in the ashes will blanket blue photons with wavelengths $\lesssim$5000\,\AA\ \citep{Kromer_DD_2010}, the duration of which depends on the mass of the He-shell. For shells that are sufficiently thick, \citet{Boyle2017_Helium} suggest that the unburnt helium could provide an observational signal in near-infrared (NIR) spectra, and for those with a low progenitor mass ($\lesssim$1.0\,$\Msun$), \citet{polin_nebular_2021} predict significant [\ion{Ca}{2}] emission in the nebular phase of the SNe.

The He-shell DDet scenario could naturally account for the observational diversity in the SN\,Ia population. Using different sets of He-shell mass and C/O core mass, one can reproduce a variety of observables in ``normal'' SNe\,Ia with typical luminosities and spectral features near peak light \citep[e.g.,][]{polin_observational_2019,Shen_2D_2021}, or peculiar subluminous ones \citep[e.g.,][]{polin_observational_2019}. 

For the He-shell DDet SNe that show ``normal'' characteristics near peak brightness, the mass of the C/O core should be $\gtrsim$1\,$\Msun$, and the mass of the He-shell is expected to be low \citep[$\lesssim$0.03\,$\Msun$;][]{polin_observational_2019,Magee_2021,Shen_2D_2021}. Recently, it was reported that SN\,2018aoz \citep{Ni_2022}, an SN\,Ia showing a rapid redward color evolution within $\sim$12\,hr after first light, could be explained by a sub-\Mch\ DDet model (a 1.05\,$\Msun$ C/O core and a 0.03\,$\Msun$ He-shell). After this red excess, the photometric evolution is consistent with that of normal SNe\,Ia, when the ashes of the thin He-shell become optically thin. However, some of its peak-time and nebular-phase spectral properties are not consistent with a He-shell DDet scenario \citep{Ni_2022b}, making its nature debatable. To date, only a small fraction of SNe\,Ia have been discovered sufficiently early for possible detection of early flashes \citep[e.g.,][]{Deckers_2022}. While there could be a large underlying population of normal SNe\,Ia triggered by He-shell DDet, currently it is hard to verify this scenario. 

In contrast, if the He-shell is sufficiently massive, such that the ashes of the shell remain optically thick over a much more extended time, the SN could appear unusually red even near peak light. Such peculiar SNe\,Ia could be normal in brightness, SN\,2016jhr being the only event ever reported \citep{jiang_16jhr_2017}, which shows a normal peak brightness ($M_B\approx-18.8$\,mag), but exhibits an early red flash and keeps a red $g-r$ color throughout its evolution. Its photometric evolution as well as around-maximum spectrum could be explained by a near-\Mch\ DDet model. %(a 1.38\,$\Msun$ C/O core and a 0.03\,$\Msun$ He-shell). 
On the other hand, events with a total progenitor mass $<$1\,$\Msun$ would be subluminous. SN\,2018byg \citep{de_18byg_2019} is a prototype of this subclass. During the late stages of preparing this paper, \citet{Dong_16dsg_2022} presented another thick He-shell DDet candidate, SN\,2016dsg, accompanied with an archival transient OGLE-2013-SN-079 \citep{Inserra_OGLE13_079_2015}. All three events are faint, red, and showing strong line-blanketing in maximum-light spectra. A tentative detection of unburnt helium in SN\,2016dsg was also reported by \citet{Dong_16dsg_2022}. We refer to these peculiar events as thick He-shell DDet SNe, which best characterize the physics leading to their peculiarities, with the caveat that the threshold of a He-shell being ``thick'' depends on the core mass. The small sample size to date suggests that thick He-shell events might be intrinsically rare.

It has been proposed that some, if not all, of the calcium-rich (Ca-rich) gap transients, a population of faint SNe with conspicuous [\ion{Ca}{2}] emission in the nebular phase \citep{Filippenko_2003, Kasliwal_2012}, also arise from He-shell DDet \citep{Dessart_2015,de_Ca_rich_2020,polin_nebular_2021}. A subclass of Ca-rich transients resemble SNe\,Ia near peak light (termed Ca-Ia objects), marked by strong \ion{Si}{2} absorption and the absence of optical \ion{He}{1} lines. There are only three Ca-Ia objects \citep[PTF\,09dav, SN\,2016hnk, and SN\,2019ofm;][]{de_Ca_rich_2020}, all showing mild to strong line-blanketing in spectra, and hence could be He-shell DDet objects \citep[e.g.,][]{jacobson-galan_16hnk_2020}. Nonetheless, they also exhibit properties similar to those of other types of subluminous SNe\,Ia, such as the strong \ion{O}{1} absorption widely seen in SN\,1991bg-like \citep[91bg-like;][]{Filippenko_91bg_1992} objects  but not prominent in other He-shell DDet candidates. PTF\,09dav shows the weakest line-blanketing among the three and exhibits features that are attributed to some rare elements such as \ion{Sc}{2} \citep{Sullivan_2011}, which cannot be immediately explained by either He-shell DDet or deflagration models. SN\,2016hnk could also be explained by the deflagration of a near-\Mch\ WD \citep{galbany_16hnk_2019}. In summary, the nature of Ca-Ia objects remains ambiguous.

In this paper, we present observations of another promising thick He-shell DDet candidate, \sn. This peculiar SN\,Ia highly resembles SN\,2018byg in photometric and spectroscopic properties, and exhibits a remarkable feature in the NIR spectrum that could be attributed to the unburnt helium. In Section~\ref{sec:obs}, we report the observations of \sn, which are analyzed in Section~\ref{sec:analysis}, where we show its similarities with other He-shell DDet SNe and discuss the tentative \ion{He}{1} absorption features. We use a grid of He-shell DDet models to fit the data of \sn, and present the results in Section~\ref{sec:model}. Then we expand our discussion to other He-shell DDet SNe, discussing the possibly ubiquitous absorption features in their NIR spectra near 1\,\micron\ (Section~\ref{sec:1um}) and their diversity in host environments (Section~\ref{sec:host}). We draw our conclusions in Section~\ref{sec:conclusion}.

\section{Observations} \label{sec:obs}
\subsection{Discovery}

\sn\ was first discovered by the Zwicky Transient Facility \citep[ZTF;][]{Bellm_ZTF_2019a,Graham_ZTF_2019, Dekany_ZTF_2020} on 2020 May 03.463 (UT dates are used throughout this paper; MJD 58972.463) with the 48-inch Samuel Oschin Telescope (P48) at Palomar Observatory. The automated ZTF discovery pipeline \citep{Masci_ZTF_2019} detected \sn\ using the image-differencing technique of \citet{Zackay_imagesub_2016}. The candidate passed internal thresholds \citep[e.g.,][]{Mahabal_ZTFML_2019, Duev_ZTFML_2019}, leading to the production and dissemination of a real-time alert \citep{Patterson_ZTFalert_2019} and the internal designation ZTF20aayhacx. It was detected with $g_\mathrm{ZTF} = 19.86 \pm 0.15\,$mag at $\alpha_\mathrm{J2000}=17^\mathrm{h}53^\mathrm{m}12^\mathrm{s}.651$, $\delta_\mathrm{J2000}=-00^\circ51'21\farcs{81}$ and announced to the public by \citet{Fremling_report_2020}. The host galaxy, PSO J175312.663+005122.078, is a dwarf galaxy, to which \sn\ has a projected offset of only $0\farcs3$. The last nondetection limits the brightness to $r_\mathrm{ZTF} > 20.7$\,mag on 2020 April 27.477 (MJD 58966.477; 5.99\,days before the first detection). This transient was classified as an SN\,Ia by \citet{TNS_2020}. This classification was confirmed with the \texttt{SuperNova IDentification} \citep[\texttt{SNID};][]{Blondin_SNID_2007} code, showing \sn\ is more consistent with SNe Ia. Templates of other hydrogen-poor SNe, including Type Ib and Type Ic SNe, do not match the spectra of \sn.

\subsection{Host Galaxy Observations}
On 2022 March 31, two years after the transient faded, we took a spectrum of its host galaxy using the DEep Imaging Multi-Object Spectrograph (DEIMOS) on the Keck-II 10\,m telescope \citep{DEIMOS_2003}, with a total integration time of 3200\,s. It was reduced with the \texttt{PypeIt} Python package \citep{pypeit:joss_pub} and is displayed in Figure~\ref{fig:host_spec}. 
The host exhibits strong, narrow emission lines including H$\alpha$, H$\beta$, [\ion{N}{2}] $\lambda\lambda$6548, 6583, [\ion{O}{3}] $\lambda\lambda$4959, 5007, and [\ion{S}{2}] $\lambda\lambda$6716, 6731. By fitting all these emission features with Gaussian profiles, we obtain an average redshift of $z=0.0309\pm0.0003$. With the diagnostic emission line equivalent width (EW) ratios ($\log$~[\ion{N}{2}]/H$\alpha=-1.05\pm0.08$ and $\log$~[\ion{O}{3}]/H$\beta=0.19\pm0.02$)\footnote{Here [\ion{N}{2}] denotes the EW of the [\ion{N}{2}] $\lambda$6583 line, and [\ion{O}{3}] denotes the EW of the [\ion{O}{3}] $\lambda$5007 line.}, the host is consistent with star-forming galaxies in the \citet[][hereafter BPT]{BPT_1981} diagram \citep[see also][]{Veilleux_1987}. 

To estimate the distance modulus of \sn, we first use the 2M++ model \citep{Carrick2015_2M++} to obtain the peculiar velocity toward its host galaxy, PSO\,J175312.663+005122.078, to be $179\pm250$\,\kms. This, combined with the recession velocity in the frame of the cosmic microwave background\footnote{\url{https://ned.ipac.caltech.edu/velocity_calculator}} (CMB) $v_\mathrm{CMB}=9136$\,\kms, yields a net Hubble recession velocity of $9307\pm250$\,\kms. Adopting H$_0=70$\,\kms\,Mpc$^{-1}$, $\Omega_M=0.3$, and $\Omega_\Lambda=0.7$, we estimate the luminosity distance of \sn\ to be 136.1\,Mpc, equivalent to a distance modulus of $\mu=35.67\pm0.06$\,mag.

To evaluate the potential host galaxy extinction, we estimate the H$\alpha$ to H$\beta$ flux ratio to $3.26\pm0.13$, while the theoretical, extinction-free value is 2.86 \citep[assuming case B recombination;][]{Osterbrock_2006}. Using the extinction law from \citet{Fitzpatrick1999} and assuming $R_V=3.1$, this yields $E(B-V)_\mathrm{host}=0.11\pm0.04$\,mag. It is consistent our SED fitting result (illustrated in Section~\ref{sec:host}), $E(B-V)_\mathrm{host}=0.13\pm0.01$\,mag, and sets an upper limit of the actual amount of reddening \sn\ suffers in its host galaxy.


\subsection{Optical Photometry}
\sn\ was monitored in the $g_\mathrm{ZTF}$ and $r_\mathrm{ZTF}$ bands by ZTF as part of its ongoing Northern Sky Survey \citep{Bellm_ZTF_2019b}. We adopt a Galactic extinction of $E(B-V)=0.404\,$mag \citep{Schlafly2011}, and correct all photometry using the \citet{Fitzpatrick1999} extinction model. The host extinction is not well constrained. While the potential host extinction could be up to $E(B-V)_\mathrm{host}\approx0.13$, the lack of \ion{Na}{1}\,D absorption at the redshift of the host galaxy is consistent with no additional host extinction, though see \citet{Poznanski_2011} for caveats on the use of \ion{Na}{1}\,D absorption as a proxy for extinction. Thus throughout the paper, we adopt a fiducial assumption of no host extinction and discuss the possible effect of addition extinction. Unless otherwise specified, the data displayed in the figures are only corrected for Galactic extinction.

The forced-photometry absolute light curves\footnote{\url{https://web.ipac.caltech.edu/staff/fmasci/ztf/forcedphot.pdf}} in $g_\mathrm{ZTF}$ and $r_\mathrm{ZTF}$ are shown in Figure~\ref{fig:photometry}, where we display all measurements having a signal-to-noise ratio (SNR) greater than 2. The light curves are reduced using the pipeline from Miller et al. (2022, in preparation); see also \citet{Yao_2019}.

\begin{figure*}
    \centering
    \includegraphics[width=\textwidth]{photometry.pdf}
    \caption{Comparison of the photometric properties of \sn\ with those of SN\,2011fe \citep[normal SN\,Ia;][]{Pereira_2013}, SN\,2016jhr \citep[normal-luminosity He-shell DDet;][]{jiang_16jhr_2017}, and SN\,2018byg \citep[subluminous He-shell DDet;][]{de_18byg_2019}. \textit{Left}: Multiband light curves. The upper (lower) panel shows the evolution in the $r$-band ($g$-band) absolute magnitude. \textit{Right}: $g-r$ color evolution. For each object, the peak luminosity epoch is marked by a vertical line with the corresponding color on the bottom axis. The gray circles denote the $g_\mathrm{ZTF}-r_\mathrm{ZTF}$ color evolution of 62 normal SNe\,Ia (open circles) with prompt observations within 5\,days of first light by ZTF \citep{Bulla2020}. }
    \label{fig:photometry}
\end{figure*}

\subsection{Optical Spectroscopy}\label{sec:optical_spec}
We obtained optical spectra of the object from $\sim$$-10$\,days to $\sim$+150\,days relative to the $r_\mathrm{ZTF}$-band peak, using the Spectral Energy Distribution Machine \citep[SEDM;][]{SEDM_2018} on the automated 60\,inch telescope \citep[P60;][]{P60_2006} at Palomar Observatory, the Kast Double Spectrograph \citep{miller1994kast} on the Shane 3\,m telescope at Lick Observatory, the Andalucia Faint Object Spectrograph and Camera (ALFOSC)\footnote{\url{https://www.not.iac.es/instruments/alfosc/}} installed at the Nordic Optical Telescope (NOT), the Double Beam Spectrograph (DBSP) on the 200\,inch Hale telescope \citep[P200;][]{P200_1982}, and the Low Resolution Imaging Spectrometer (LRIS) on the Keck-I 10\,m telescope \citep{Keck_1995}. With the exception of observations obtained with SEDM, all spectra were reduced using standard procedures \citep[e.g.,][]{Matheson_2000}. The SEDM spectra were reduced using the custom \texttt{pysedm} software package \citep{Rigault_pysedm_2019}. Details of the spectroscopic observations are listed in Table~\ref{tab:spec}, and the resulting spectral sequence is shown in Figure~\ref{fig:spec_evo}. All the spectra listed in Table~\ref{tab:spec} will be available on WISeREP \citep{wiserep_2012}.

\input{./tables/spec.tex}
\begin{figure*}
    \centering
    \includegraphics[width=\textwidth]{optical_spec_evolution.pdf}
    \caption{Optical spectra of \sn, typical of a peculiar SN\,Ia triggered by He-shell DDet. \textit{Left}: optical spectral sequence of \sn. Rest-frame phases (days) relative to the $r_\mathrm{ZTF}$-band peak and instruments used are posted next to each spectrum. The spectra after Galactic extinction correction are shown in gray. The black lines are binned spectra with a bin size of 10\,\AA, except for the SEDM spectra, whose resolution is lower than the bin size. In the last two spectra, we have subtracted the light from the host galaxy. Only regions with $\mathrm{SNR}>2.5$ after binning are plotted. The corresponding wavelengths of the \ion{Si}{2} $\lambda$6355 line (with an expansion velocity of 10,000\,\kms) and the \ion{Ca}{2} IRT (with expansion velocities of both 10,000\,\kms and 25,000\,\kms) are marked by the vertical dashed lines.
    \textit{Right}: spectral comparison with SN\,2018byg \citep[subluminous He-shell DDet;][]{de_18byg_2019} and SN\,2004da \citep[normal luminosity;][]{Silverman_2012}.}
    \label{fig:spec_evo}
\end{figure*}

\subsection{Near-Infrared Spectroscopy}
We obtained one NIR (0.8--2.5\,\micron) spectrum of \sn\ using the Gemini near-infrared spectrometer \citep[GNIRS;][]{GNIRS1998} on the Gemini North telescope on 2020 June 9 ($\sim$22\,days after $r_\mathrm{ZTF}$-band peak), with an integration time of 2400\,s. The GNIRS spectrum was reduced with \texttt{PypeIt}.

\begin{figure*}
    \centering
    \includegraphics[width=\textwidth]{NIR_spec.pdf}
    \caption{NIR spectra of \sn\ and two normal-luminosity SNe\,Ia, SN\,2004ab and SN\,2004da \citep{Marion2009_NIR}, both showing highly similar spectral features except the absorption line near 1\,\micron. All spectra were obtained at similar phases. For each spectrum, the continuum at $\gtrsim$1.2\,\micron\ is significantly reshaped by the line-blanketing from Fe-group elements (red stripes), which are continuous emission features composed of unresolved Fe-group lines peaking at $\sim$1.30, 1.55, 1.75, 2.00, 2.10, and 2.25\,\micron\ \citep{Marion2009_NIR}. Between these peaks lie multiple strong \ion{Co}{2} absorption lines (blue stripes), for which a typical post-maximum expansion velocity of 8000\,\kms\ is assumed. The purple stripes correspond to \ion{Fe}{2} $\lambda$9998 and \ion{Fe}{2} $\lambda$10500, also with an expansion velocity of 8000\,\kms.}
    \label{fig:NIR_spec}
\end{figure*}

\section{Analysis} \label{sec:analysis}
\subsection{Photometric Properties} \label{sec:phot_analysis}
\sn\ exhibited a fainter light curve than normal SNe\,Ia. In Figure~\ref{fig:photometry}, we compare the photometric properties of \sn\ with the nearby, well-observed SN\,2011fe in $g_\mathrm{ZTF}$ and $r_\mathrm{ZTF}$ synthetic photometry from the spectrophotometric time series of \citet{Pereira_2013}, as well as two He-shell DDet candidates, including the normal-luminosity thin He-shell candidate SN\,2016jhr \citep{jiang_16jhr_2017} and the subluminous thick He-shell candidate SN\,2018byg \citep{de_18byg_2019}. All of these light curves have been corrected for Galactic reddening, while $K$-corrections have not been performed\footnote{These SNe were all observed in slightly different $g$ and $r$ filters.}, because we do not have complete spectral sequences of these peculiar events.

While the observational coverage is sparse in the rise to maximum light, from Figure~\ref{fig:photometry} it is clear that \sn\ is less luminous than normal SNe\,Ia (e.g., SN\,2011fe). Assuming maximum host extinction, \sn\ would be $\sim$0.3\,mag brighter in the $r_\mathrm{ZTF}$ band and $\sim$0.5\,mag in the $g_\mathrm{ZTF}$ band, making it comparable to SN\,2011fe in $r_\mathrm{ZTF}$, yet still $\sim$0.5\,mag fainter in $g_\mathrm{ZTF}$. Furthermore, there is a flatter evolution in $r_\mathrm{ZTF}$ between $-14$\,days and maximum light for both \sn\ and SN\,2018byg than there is for SN\,2011fe.  

In the right panel of Figure~\ref{fig:photometry}, we compare the color evolution ($g-r$) of these objects relative to the measured time of first light \tfl, accompanied by 62 normal SNe\,Ia (open circles) observed within 5 days of \tfl\ by ZTF \citep[from][]{Bulla2020} without $K$-correction for consistency. For \sn, the early rise of the light curve was not well sampled, so we estimate \tfl\ as the midpoint of the first detection and the last nondetection. We adopt an uncertainty in this estimate of 3\,days. All three He-shell DDet candidates are undoubtedly redder than normal SNe\,Ia. At peak light, \sn\ ($g_\mathrm{ZTF}-r_\mathrm{ZTF}\approx0.4$\,mag) was not as red as the extreme case, SN\,2018byg ($g-r\approx2.2$\,mag), but exhibited a similar color as SN\,2016jhr ($g-r\approx0.3$\,mag). Even assuming maximum host extinction, \sn\ still has a relatively red color ($g_\mathrm{ZTF}-r_\mathrm{ZTF}\approx0.2$\,mag) compared to normal SNe Ia ($g_\mathrm{ZTF}-r_\mathrm{ZTF}\approx-0.1$\,mag).

Interestingly, for both SN\,2018byg and \sn, near their maximum light the spectra sharply peak at $\sim$5200\,\r{A} in the host galaxy rest frame (see Figure~\ref{fig:spec_evo}), which is close to the red edge of the $g$ filter ($\sim$4000--5500\,\r{A}). Thus, modest redshifts ($z\gtrsim0.03$) can produce significant $K$-corrections, which constitute a substantial fraction of the observed red $g-r$ colors for this events. For \sn, using the ALFOSC spectrum obtained at $-$4\,days, we estimate the $K$-correction to be $K_{g-r}\approx -0.2$\,mag, the $g-r$ color being bluer in the rest frame. SN\,2018byg is at a higher redshift ($z=0.066$) so the $K$-correction is more extreme ($K_{g-r}\approx-1.0$\,mag). Future efforts to identify additional subluminous He-shell DDet candidates should utilize these red $g-r$ colors to improve their search efficiency.

\subsection{Optical Spectral Properties}
In Figure~\ref{fig:spec_evo}, we show the optical spectral sequence of \sn, and compare its spectra with those of some other SNe\,Ia at similar phases relative to peak brightness. For the spectra obtained after +100\,days there is clear contamination from the host galaxy, including the presence of narrow emission lines. For these spectra we subtract the galaxy light as measured in the DEIMOS spectrum from 2022 (see Section~\ref{sec:optical_spec}). The earliest spectrum was obtained by SEDM $\sim$10\,days before $r_\mathrm{ZTF}$-band peak. We only show portions of the spectrum where the $\mathrm{SNR}>2.5$. The continuum is almost featureless with some marginal detection of the \ion{Si}{2} $\lambda$6355 at $\sim$6100\,\AA, the trademark of SNe\,Ia. In subsequent spectra the \ion{Si}{2} features become more prominent and are clearly detected until $\sim$12\,days after peak light. We measure \ion{Si}{2} expansion velocities following a procedure similar to that of  \citet{Childress_2013,Childress_2014} and \citet{Maguire_2014}. The fitting region is selected by visual inspection. The continuum is assumed to be linear, and the absorption profile after the continuum normalization is assumed to be composed of double Gaussian profiles centered at 6347\,\AA\ and 6371\,\AA. Within the model, the continuum flux density at the blue and red edges are free parameters for which we adopt a normal distribution as a prior. The mean and standard deviation for the distribution are the observed flux density and its uncertainty (respectively) at each edge of the fitting region. Three more parameters (amplitude, mean velocity, logarithmic velocity dispersion) are used to characterize the double-Gaussian profile, whose priors are set to be flat. This means the depths and widths of both peaks are forced to be the same, as \citet{Maguire_2014} adopted in the optically thick regime. The posteriors of the five parameters are sampled simultaneously with \texttt{emcee} \citep{emcee_2013} using the Markov chain Monte Carlo (MCMC) method. We find that the mean expansion velocity is $\sim$11,500\,\kms\ near maximum light.

\sn\ does not show any absorption features associated with \ion{O}{1} $\lambda$7774. While the low luminosity, red color, absence of hydrogen features, and star-forming host galaxy of \sn\ are reminiscent of Type Ic SNe (SNe\,Ic), which arise from stripped-envelope massive stars, SNe\,Ic usually exhibit stronger \ion{O}{1} $\lambda$7774 lines. The ratio of the relative line depths\footnote{The relative depth is defined as the absorption line depth relative to the pseudo-continuum. See \citet{Sun_2017} for more details.} between the \ion{O}{1} $\lambda$7774 line and the \ion{Si}{2} $\lambda$6355 line is expected to be greater than 1 in typical SNe\,Ic \citep{Sun_2017, Gal-Yam_2017}, thus with the absence of the \ion{O}{1} $\lambda$7774 line, \sn\ is not likely a SN\,Ic.

In many SNe\,Ia, the \ion{Ca}{2} near-infrared triplet (\ion{Ca}{2} IRT) $\lambda\lambda$8498, 8542, 8662 causes two distinct components \citep{Mazzali_2005}, which are conventionally referred to as photospheric-velocity features (PVFs) and high-velocity features (HVFs). The PVFs originate from the main line-forming region with typical photospheric (i.e., bulk ejecta) velocities, while the HVFs are blueshifted to much shorter wavelengths, indicating significantly higher (by $\gtrsim$6000\,\kms) velocities than typical PVFs \citep{Silverman_HVF_2015}. Figure~\ref{fig:spec_evo} shows that \sn\ has prominent HVFs of \ion{Ca}{2} IRT. The HVFs are visible in our first spectrum of \sn, and remain prominent through $+36$\,days. Using a similar technique in modeling the \ion{Si}{2} features, we fit the HVFs and PVFs simultaneously. Both are fit by multiple Gaussian profiles assuming each line in the triplet can be approximated by the same profile (i.e., same amplitude and velocity dispersion). A best-fit expansion velocity of HVFs is $\sim$26,000\,\kms. A clear delineation between the HVFs and PVFs is visible $\sim$4\,days before peak light. Since then we fit the broad absorption features with two different velocity components simultaneously. From $\sim$$-5$\,days to $\sim$$+20$\,days, the velocity of HVFs slightly declines but stays above $\sim$24,000\,\kms, and the velocity of PVFs declines from $\sim$11,000\,\kms\ to $\sim$9,000\,\kms. As in normal SNe\,Ia, the relative strength between the HVFs and PVFs decreases with time.

The optical spectral evolution of \sn\ resembles that of SN\,2018byg, a subluminous thick He-shell DDet SN. At early times, both SNe were relatively blue and featureless, with broad and shallow \ion{Ca}{2} IRT absorption. As they evolved closer to maximum light, they developed strong continuous absorption blueward of $\sim$5000\,\AA. After correcting the potential host reddening on \sn, this continuous absorption is still prominent. Meanwhile, \ion{Si}{2} $\lambda$6355 and the \ion{Ca}{2} IRT became more prominent. Neither \ion{O}{1} nor \ion{S}{2} was detected in either object. In the He-shell DDet scenario, a large amount of Fe-group elements would be synthesized in the shell, which would cause significant line-blanketing near maximum light \citep{Kromer_DD_2010, polin_observational_2019} and high-velocity intermediate-mass elements like \ion{Ca}{2} \citep{Fink_DD_2010, Kromer_DD_2010,Shen_DD_2014}. The similarity to SN\,2018byg makes \sn\ another promising He-shell DDet SN candidate.

SN\,2004da is a normal SN\,Ia that shows similarities to \sn\ in the NIR (Section~\ref{sec:NIR_spec}); however, the two SNe are very different in the optical (Figure~\ref{fig:spec_evo}). From this comparison it is clear that \sn\ is not a normal SN\,Ia. 

We obtained two LRIS spectra at $+117$\,days and $+152$\,days, both of which are dominated by Fe-group elements and resemble those of normal SNe Ia \citep[e.g., SN\,2011fe;][]{Mazzali_2015}, showing some enhancement in flux between $\sim$4500 and $\sim$6000\,\AA. There are no signs of emission feature related to the [\ion{Ca}{2}] $\lambda\lambda$7291, 7324 doublet.

\subsection{NIR Spectral Properties}
\label{sec:NIR_spec}
The NIR spectrum of \sn\ is compared with those of two normal SNe\,Ia at a similar phase in Figure~\ref{fig:NIR_spec} \citep[data for SN\,2004ab and SN\,2004da from][]{Marion2009_NIR}. \sn\ shows a strong absorption feature at $\sim$0.99\,\micron, which is not seen in normal SNe\,Ia. This feature was still significant two weeks later, as detected with LRIS on Keck (see Figure~\ref{fig:hvf_comp}), though it was only partially covered. Aside from this prominent feature, \sn\ resembles normal SNe\,Ia in the NIR. The shape of the continuum redward of $\sim$1.2\,\micron\ is significantly altered by line-blanketing from Fe-group elements. Just like normal SNe\,Ia, \sn\ shows an enhancement of flux at about 1.30, 1.55, 2.00, 2.10, and 2.25\,\micron, accompanied by several \ion{Co}{2} absorption lines. It is especially similar to SN\,2004da at +25\,days as the steep increase in flux at $\sim$1.55\,\micron, known as the \textit{H}-band break \citep{Hsiao_CSP_2019}, has become less prominent. To summarize, the NIR spectrum of \sn\ is dominated by Fe-group elements, consistent with the nucleosynthesis yields following a thermonuclear explosion of a C/O WD. However, the  1\,\micron\ feature adds to the peculiarities of \sn\ as a SN\,Ia, suggesting an unusual explosion mechanism.

\citet{Marion2009_NIR} presented a sample of 15 NIR spectra of normal SNe\,Ia between +14 and +75\,days relative to maximum light, and none of those spectra show prominent absorption features around 1\,\micron. We have investigated several potential identifications for this feature (see below), none of which provides a completely satisfying explanation.

The most tantalizing possibility is that the absorption is due to \ion{He}{1} $\lambda$10830. If \sn\ is a He-shell DDet SN, then unburnt helium could lead to observed absorption in the spectrum, as shown in the sub-\Mch\ He-shell DDet models of \citet{Boyle2017_Helium}. Figure~\ref{fig:hvf_comp} shows that the 1\,\micron\ feature, if associated with \ion{He}{1} $\lambda$10830, has a velocity of $\sim$26,000\,\kms. This aligns well where the helium lies in He-shell DDet models when the ejecta have reached homologous expansion \citep{Kromer_DD_2010, polin_observational_2019}, yet it is unclear whether the high-velocity unburnt helium could stay optically thick until weeks after maximum light. The \ion{Ca}{2} IRT also exhibits similarly high velocities at the same phase ($\sim$24,000\,\kms), meaning that high-velocity absorption is not impossible at this phase. The expansion velocity in the ejecta is roughly linearly proportional to the radius, so such a high velocity indicates that both the \ion{Ca}{2} IRT and the tentative \ion{He}{1} absorption line form far outside the normal photosphere, which has a velocity of only $\sim$10,000\,\kms. The two-dimensional (2-D) models of \citet{Kromer_DD_2010} also suggest that helium may expand faster than the synthesized calcium in the He-shell. In this sense, the He-shell DDet scenario is supported because any unburnt helium would be located in the outermost ejecta.

We cannot claim an unambiguous detection of \ion{He}{1}, however, as our spectra lack definitive absorption from other \ion{He}{1} features that we would expect to be prominent, such as \ion{He}{1} $\lambda$20581. Considering a line velocity of $\sim$26,000\,\kms\ and a host-galaxy redshift of 0.0309, this line will be blueshifted to $\sim$1.95\,\micron\ in the observer frame, which overlaps with some strong telluric lines within 1.8--2.0\,\micron. After telluric correction, the SNR reaches $\sim$5, with which we still cannot see any significant absorption feature. An upper limit of the equivalent width is determined to be $<$2\% that of the \ion{He}{1} $\lambda$10830 line, while theoretically, the $\lambda$20581 line is supposed to be only a factor of 6--12 weaker, depending on the temperature \citep{Marion2009_NIR}. The observed 1\,\micron\ feature in \sn\ is as strong as the \ion{He}{1} $\lambda$10830 line in many helium-rich Type Ib supernovae \citep[SNe\,Ib; see][for a review of SN spectral classification]{Filippenko97, Gal-Yam_2017}. In SNe\,Ib, the \ion{He}{1} $\lambda$20581 line is weaker than the \ion{He}{1} $\lambda$10830 line, yet still prominent \citep{CSP_Ibc_2022}. In one of the models of \citet{Boyle2017_Helium}, there is no obvious \ion{He}{1} $\lambda$20581 absorption in the synthetic spectra (see their Figure~7), but the model is intended to be representative of normal-luminosity SNe\,Ia. If the 1\,\micron\ feature is associated with \ion{He}{1}, it is unusual that we do not detect a corresponding feature around 2\,\micron.

Other possible identifications for the 1\,\micron\ feature include \ion{Mg}{2} $\lambda$10927, \ion{C}{1} $\lambda$10693, and \ion{Fe}{2} $\lambda$10500 and $\lambda$10863. The \ion{Mg}{2} $\lambda$10927 line is prevalent in the NIR spectra of SNe\,Ia, but usually disappears within a week after peak brightness \citep{Marion2009_NIR}. In \sn\ the 1\,\micron\ feature was still visible more than a month after peak brightness in the Keck/LRIS spectrum. A \ion{Mg}{2} $\lambda$10927 identification would require an absorption velocity of $\sim$28,000\,\kms, $\sim$20\% faster than the HVFs of \ion{Ca}{2} IRT at the same phase. Such a high-velocity \ion{Mg}{2} line has never been seen in other SNe\,Ia, and requires a high magnesium abundance in the outermost ejecta. However, the amount of magnesium synthesized in the detonation of the He-shell is expected to be tiny \citep{Fink_DD_2010,Kromer_DD_2010,polin_observational_2019,polin_nebular_2021}. On the other hand, if we attribute this 1\,\micron\ feature to high-velocity \ion{Mg}{2}, we would expect an even stronger \ion{Mg}{2} $\lambda$9227 line to be blueshifted to the red edge of the \ion{Ca}{2} IRT, which is not detected. Given the strength of the 1\,\micron\ feature, the \ion{Mg}{2} $\lambda$9227 line should not be completely obscured by the \ion{Ca}{2} IRT features.

\ion{C}{1} $\lambda$10693 is not observed as frequently as \ion{Mg}{2} $\lambda$10927 in SNe\,Ia. \citet{Hsiao_CSP_2019} presented a sample of five SNe\,Ia with \ion{C}{1} detections, showing that the \ion{C}{1} feature is strongest for fainter, fast-declining objects. However, in their sample, the \ion{C}{1} line is a pre-maximum feature which fades away as the luminosity peaks, so the discrepancy in phase is large. The required expansion velocity $\sim$22,000\,\kms\ is substantially faster than the estimated carbon velocity for the sample of \citet{Hsiao_CSP_2019} ($\sim$10,000--12,000\,\kms), but still consistent with the HVFs of \ion{Ca}{2} IRT in \sn. Nonetheless, no significant carbon absorption is detected in the optical. It is also noteworthy that the amount of unburnt carbon is expected to be minimal in sub-\Mch WDs ignited by detonation \citep{polin_observational_2019}, in contrast to near-\Mch\ WDs ignited by pure deflagration where the carbon burning could be incomplete. We therefore would not expect to detect any carbon features in a He-shell DDet SN.

The \ion{Fe}{2} features in SNe\,Ia usually start to develop approximately three weeks after peak brightness, which is about the same phase as we obtained our GNIRS spectrum. Two \ion{Fe}{2} lines, $\lambda$9998 and $\lambda$10500, are actually visible on the blue/red wings of the 1\,\micron\ feature (see Figure~\ref{fig:NIR_spec}). The \ion{Fe}{2} $\lambda$10863 line is not detected in the GNIRS spectrum. SN\,2004da shows very similar \ion{Fe}{2} features near 1\,\micron, in which \ion{Fe}{2} $\lambda$10500 is the strongest line ate this phase, as displayed in Figure~\ref{fig:NIR_spec}. They correspond to an expansion velocity of $\sim$8000\,\kms, which is consistent with the PVFs of the \ion{Ca}{2} IRT at the same epoch. They also match the same two lines for normal SNe\,Ia \citep{Marion2009_NIR}, making the identification more reliable. Obviously, these two \ion{Fe}{2} features are wider and shallower than the strong feature between them. We fit the 1\,\micron\ feature with three Gaussian profiles. Two of them are set to be the blueshifted \ion{Fe}{2} $\lambda$9998 and $\lambda$10500, and the other is an uncorrelated Gaussian profile which mainly describes the deep absorption feature in the center of the line complex. We find that the shallower and wider \ion{Fe}{2} lines only make up $\sim$40\% of the total equivalent width, and the remaining $\sim$60\% comes from the central feature, which cannot be accounted for by any \ion{Fe}{2} feature at the same velocity. Given the similarity of the Fe-group line-blanketing between the GNIRS spectrum with the spectrum of SN\,2004da at +25\,days, the distribution of Fe-group elements inside each SN ejecta should be somewhat similar, so the central region of the 1\,\micron\ feature is not likely to be associated with \ion{Fe}{2} either.

\section{Discussion} \label{sec:discussion}
\subsection{Models} \label{sec:model}
\begin{figure*}
    \centering
    \includegraphics[width=\textwidth]{model.pdf}
    \caption{Spectrophotometric comparison of \sn\ observations with two thick He-shell DDet models from \citet{polin_observational_2019}. For the $0.82\,\Msun+0.13\,\Msun$ model, only the Galactic extinction of $E(B-V)=0.404$ is applied to the synthetic spectra and photometry; for the $0.87\,\Msun+0.13\,\Msun$ model, additional reddening of $E(B-V)_\mathrm{host}=0.13$ from the host galaxy is assumed. {\it Left:} Comparison of the ZTF photometry with the synthetic light curves. The model parameters are indicated in the legend as (C/O core mass $+$ He-shell mass). The upper (lower) panel shows the evolution in $r_\mathrm{ZTF}$ ($g_\mathrm{ZTF}$). The phases have been rescaled to the host-galaxy rest frame. {\it Right:} Comparison of the observed spectra with the models around peak luminosity. The shaded regions correspond to the coverage of the ZTF $g$ and $r$ filters with transmission above half maximum. All spectra are normalized such that they would have a same synthetic brightness in $r_\mathrm{ZTF}$. The synthetic spectra are further binned with a size of 20\,\AA.}
    \label{fig:model}
\end{figure*}

We model \sn\ using the methods outlined by \citet{polin_observational_2019}; the process is twofold. After choosing an initial model that describes a WD of a given mass with a choice of He-shell mass, we use the \texttt{CASTRO} code \citep{Almgren_Castro_2010} to perform a 1-D hydrodynamic simulation with simultaneous nucleosynthesis from the time of He-shell ignition through the secondary detonation and until the ejecta have reached homologous expansion ($\sim$10\,s). At this point we take the ejecta profile (velocity, density, temperature and composition) and use the Monte Carlo radiative transport code \texttt{SEDONA} \citep{Kasen_Sedona_2006} to calculate synthetic light curves and spectra of our model under the assumption of local thermal equilibrium (LTE). 

For He-shell DDet SNe, the peak luminosity in $r_\mathrm{ZTF}$ is a proxy of the amount of \Ni\ synthesized in the detonation, which reflects the total progenitor mass \citep[C/O core $+$ He-shell;][]{polin_observational_2019}. We find models with a total mass of $0.95\,\Msun$ reproduce the $r_\mathrm{ZTF}$-band peak brightness well if there is no extinction from the host galaxy. By assuming maximum host extinction, $E(B-V)_\mathrm{host}=0.13$, the intrinsic luminosity in $r_\mathrm{ZTF}$ would be $\sim$25\% higher, and the corresponding progenitor mass would be roughly $1.00\,\Msun$. The uncertainty in the extinction thus limits the possible precision to which the progenitor mass of \sn\ can be constrained.

Nonetheless, the major photometric and spectroscopic features of \sn\ are consistent with those of a DDet SN with a massive shell. In Figure~\ref{fig:model}, we show the comparison of the observations of \sn\ with thick He-shell ($0.13\,\Msun$) DDet models with total masses of $0.95\,\Msun$ and $1.00\,\Msun$, respectively. The synthetic spectra extracted from model are manually reddened to mimic the Galactic extinction ($E(B-V)=0.404$), while additional host extinction of $E(B-V)_\mathrm{host}=0.13$ is added to the $0.87\,\Msun+0.13\,\Msun$ model.
Both models reproduce the overall evolution in $r_\mathrm{ZTF}$, but fail to provide a reasonable fit to the light curve in $g_\mathrm{ZTF}$. Specifically, the peak brightness in $g_\mathrm{ZTF}$ is overestimated in the $0.87\,\Msun+0.13\,\Msun$ model but underestimated in the $0.82\,\Msun+0.13\,\Msun$ model. Then after the peak light in $r_\mathrm{ZTF}$, both models show a rapidly declining $g_\mathrm{ZTF}$-band brightness, which is $\gtrsim$1\,mag fainter than our observations at the same epoch. 

The spectral comparison reveals more details. We find that both models, especially the $0.82\,\Msun+0.13\,\Msun$ one, decently fit the ALFOSC spectrum obtained $\sim$4\,days prior to the peak at rest-frame wavelengths $\gtrsim$5500\,\r{A}, including the absorption features (\ion{Si}{2} $\lambda$6355 and both velocity components of \ion{Ca}{2} IRT) and the continuum. The same is true for the SEDM spectrum obtained $\sim$4\,days after peak light, though both models overpredict flux excess in the \ion{Ca}{2} IRT P-Cygni profile. 
Meanwhile, both models provide a poor fit to the observation in bluer regions. Before peak light, the $0.87\,\Msun+0.13\,\Msun$ model exhibits weaker Fe-group line-blanketing, thus showing a much higher total flux in $g_\mathrm{ZTF}$. The $0.82\,\Msun+0.13\,\Msun$ model provides a proper level of line-blanketing, but the continuous absorption in the synthetic spectrum terminates at a longer wavelength ($\sim$5400\,\r{A}, as opposed to $\sim$5200\,\r{A} in the ALFOSC spectrum). As we have already mentioned in Section~\ref{sec:phot_analysis} when discussing $K$-corrections, the observed flux in $g_\mathrm{ZTF}$ is extremely sensitive to the red edge of the the line-blanketing region, which, in the observer frame, is close to the edge of the filter. Figure~\ref{fig:model} shows that while the actual $f_\lambda$ peaks within the $g_\mathrm{ZTF}$ filter near peak light, in the $0.82\,\Msun+0.13\,\Msun$ model the synthetic $f_\lambda$ peaks in the gap between the $g_\mathrm{ZTF}$ and $r_\mathrm{ZTF}$ filters. The same is true for the $0.87\,\Msun+0.13\,\Msun$ model after the $r_\mathrm{ZTF}$-band peak. Interestingly, this mismatch is also seen when fitting similar DDet models to SN\,2018byg \citep[][see Figure~6]{de_18byg_2019} despite the convincing fit to observations in longer wavelengths, suggesting this is one of the systematics in our models. By manually shifting the synthetic spectra at $-$4\,days in the $0.82\,\Msun+0.13\,\Msun$ model blueward by 200\,\r{A}, we find that the corresponding synthetic magnitude in $g_\mathrm{ZTF}$ immediately increases by $\sim$0.5\,mag. Given the sensitivity of brightness in $g_\mathrm{ZTF}$ on modeling the line-blanketing and the uncertainty in our models from \citet{polin_observational_2019}, we do not attempt to fit the $g_\mathrm{ZTF}$-band light curve of \sn\ even near its peak light.

The systematics in modeling the line-blanketing (and the flux in many similar $g$ bands) may be attributed to a variety of factors on handling the explosion and radiative transfer. First, throughout the simulations we assume local thermodynamic equilibrium (LTE), which is not valid once the ejecta become optically thin. Typically the bulk ejecta of a sub-\Mch\ SN\,Ia remain optically thick for $\sim$30\,days after the explosion. But in modeling the $g_\mathrm{ZTF}$-band brightness, the LTE assumption is trickier because the major opacity in $g_\mathrm{ZTF}$ comes from the Fe-group line-blanketing in the outermost ejecta, where the optical depth may evolve differently from that near the photosphere. Hence, the LTE condition may become inapplicable much earlier. Furthermore, our 1-D He-shell model is not capable of capturing multidimensional effects in the explosion such as asymmetries. The viewing angle is known to have a significant influence on the observed light curves \citep{Kromer_DD_2010, Sim_2012, Gronow_2020, Shen_2D_2021}, especially in bluer bands where the line-blanketing depends sensitively on the distribution of He-shell ashes \citep{Shen_2D_2021}. In previous studies of other He-shell DDet objects, the $g$-band brightness is systematically underpredicted shortly after the peak, despite the fact that redder bands can be fit decently \citep[e.g.,][]{jiang_16jhr_2017,jacobson-galan_16hnk_2020}.

Another discrepancy occurs in the late-time spectra. It is argued in \citet{polin_nebular_2021} that as the total progenitor mass in the He-shell DDet decreases, the SN gets fainter and the major coolants in the nebular phase change smoothly from Fe-group elements to the [\ion{Ca}{2}] $\lambda\lambda$7291, 7324 doublet, which is also a hallmark for Ca-rich transients and is prominent in the Ca-Ia objects \citep{galbany_16hnk_2019,De_Ca-rich_2020}. For a total progenitor mass $\lesssim$1.0\,$\Msun$, [\ion{Ca}{2}] emission features are expected to dominate Fe-group features, clearly in contrast to what we see in \sn. This suggests that either \sn\ suffers a substantial amount of host reddening, such that the actual progenitor mass is $\gtrsim$1.0\,$\Msun$; or the transition between the Fe-strong and Ca-strong regimes occurs for a lower progenitor mass than simulations have predicted. We note that no significant [\ion{Ca}{2}] features are detected in SN\,2018byg and SN\,2016dsg either. Both objects are consistent with a lower progenitor mass ($\sim$0.9\,$\Msun$) using He-shell DDet models from \citet{polin_observational_2019}. Nonetheless, the last spectra of both objects were obtained at $\sim$$+50$\,days. Even for subluminous objects with low ejecta masses, the ejecta might still be in the transitional era before entering the true nebular phase. The Ca-Ia object SN\,2016hnk is estimated to have an even lower progenitor mass \citep[$\sim$0.87\,$\Msun$;][]{jacobson-galan_16hnk_2020} and shows [\ion{Ca}{2}] lines indisputably, drawing a lower limit of the progenitor mass for this transition \citep[c.f.,][for discussion on the potential host galaxy extinction on SN\,2016hnk]{galbany_16hnk_2019}.

Given the strong match in the $r_\mathrm{ZTF}$-band light curves and the near-peak spectra at wavelengths $\gtrsim$5500\,\r{A} between the observations of \sn\ and the He-shell DDet models of \citet{polin_observational_2019}, we conclude that \sn\ is consistent with a DDet event ignited by a massive He-shell. The readers are referred to our Appendix~\ref{app1} where we show the comparison of our observations to models with a variety of shell masses, in which the thinner-shell models (shell masses $<$0.1\,$\Msun$) cannot reproduce the properties of \sn. Depending on the extinction in the host galaxy, the total mass of the progenitor should be $0.95$--$1.00\,\Msun$. To constrain the progenitor masses of additional He-shell DDet SNe to a higher precision, one should thoroughly discuss any potential host extinction. Multidimensional simulations with more realistic radiative transfer setups are necessary to resolve the systematics in our current models.

\subsection{The 1\,\micron\ Feature} \label{sec:1um}
\begin{figure}
    \centering
    \includegraphics[width=\linewidth]{NIR_spec_comp.pdf}
    \caption{NIR spectra of normal SNe\,Ia SN\,2011fe \citep{Mazzali_2014} and SN\,2004da \citep{Marion2009_NIR} and three subluminous SNe\,Ia  as He-shell DDet candidates --- SN\,2016hnk \citep{galbany_16hnk_2019}, SN\,2018byg \citep{de_18byg_2019}, and \sn\ (this work). All three He-shell DDet candidates show prominent absorption near 1\,\micron. The spectrum of SN\,2018byg is originally noisy, so it is binned with a size of 10\,\AA.}
    \label{fig:NIR_comp}
\end{figure}

\begin{figure*}
    \centering
    \includegraphics[width=\textwidth]{CaII_HeI_hvf.pdf}
    \caption{Spectra of \sn, SN\,2018byg \citep{de_18byg_2019}, and SN\,2016hnk \citep{galbany_16hnk_2019} in velocity space, showing the similarity in expansion velocities of the 1\,\micron\ features (lower panels) with the \ion{Ca}{2} IRT absorption features (upper panels), assuming the 1\,\micron\ features are associated with \ion{He}{1} $\lambda$10830. The red dashed lines mark the minimum of each 1\,\micron\ feature, which are displayed to guide the eye.}
    \label{fig:hvf_comp}
\end{figure*}

While the nature of the 1\,\micron\ feature remains uncertain, other He-shell DDet candidates show similar complexity in this region. In the currently small sample, only three objects (SN\,2016hnk, SN\,2018byg, and \sn) have at least one available NIR spectrum (all obtained at different phases), each exhibiting strong absorption features near 1\,\micron, as shown in Figure~\ref{fig:NIR_comp}. SN\,2016hnk has two deep absorption features at $\sim$1.02\,\micron\ and $\sim$1.17\,\micron; both are at a longer wavelength than the 1\,\micron\ feature in \sn. It is suggested in \citet{galbany_16hnk_2019} that both of them are caused by \ion{Fe}{2}, though they are deeper than in other SNe\,Ia. The velocity of the 1.02\,\micron\ feature is $\sim$21,000\,\kms\ assuming a \ion{He}{1} $\lambda$10830 origin, which, just like for \sn, is about the same as the HVFs of the \ion{Ca}{2} IRT (see Figure~\ref{fig:hvf_comp}). The PVFs of the \ion{Ca}{2} IRT of both SNe have a similar expansion velocity of $\sim$10,000\,\kms. Such a consistency in velocities is also seen in SN\,2018byg (see Figure~\ref{fig:hvf_comp}). The large width and low SNR for the 1\,\micron\ feature in SN\,2018byg make it difficult to determine an exact line velocity. The feature may be a mixture of several different lines in SN\,2018byg.

\citet{Dong_16dsg_2022} recently presented another thick He-shell DDet candidate, SN\,2016dsg, with an absorption line around 0.97--1.05\,\micron\ in a low-SNR NIR spectrum at $+16.6$\,days\footnote{SN\,2016dsg was discovered on the decline. The phase is relative to the discovery time.}. Assuming a \ion{He}{1} $\lambda$10830 origin, the minimum of the absorption profile (at $\sim$1.03\,\micron; see Figure~4 of \citealp{Dong_16dsg_2022}) corresponds to an expansion velocity of $\sim$15,000\,\kms. Interestingly, SN\,2016dsg shows the least prominent HVFs of \ion{Ca}{2} IRT ($v_\mathrm{SN\,2016dsg} \approx 15,000$\,\kms) among all the He-shell DDet candidates with NIR spectra. Once again, the scenario where both the unburnt helium and the high-velocity calcium are located at the outermost shell is favored.

Unfortunately, none of the spectra of SN\,2016dsg, SN\,2016hnk, or SN\,2018byg cover the 2\,\micron\ region; thus, it is not possible to identify the presence of helium decisively. But if the 1\,\micron\ features of these objects are of the same origin, they are more likely to be correlated with the high-velocity ejecta lying in the outmost region in the SNe, because at least for \sn, SN\,2016dsg, and SN\,2016hnk, the difference in their photospheric velocities cannot explain the discrepancy in their line velocities of the 1\,\micron\ feature. Then helium is still a promising candidate to cause strong absorption near 1\,\micron\ for these subluminous He-shell DDet SNe\,Ia.

In conclusion, every thick He-shell DDet candidate with available NIR spectra displays a strong absorption feature near 1\,\micron\footnote{We also note that a similar 1\,\micron\ feature is detected in another possibly relevant object, SN\,2012hn \citep{Valenti_12hn_2014}, in a NIR spectrum obtained at $+25$\,d. SN\,2012hn is a Ca-rich transient exhibiting weak \ion{Si}{2} lines and no optical helium features \citep[thus termed as a Ca-Ic object by][]{De_Ca-rich_2020}. It shows similar spectral properties (e.g., Fe-group line-blanketing) to those of the two Ca-Ia objects \citep{De_Ca-rich_2020}. This indicates a possible He-shell DDet origin of SN\,2012hn.}. 
This feature is not seen in normal SNe\,Ia. Interestingly, the available NIR spectra are all obtained at different epochs, suggesting that this feature may be long lived. If the feature is due to \ion{He}{1}, then DDet explosions exhibit a wide diversity in the expansion velocity. While it remains to be confirmed in a larger sample, we speculate that anomalously strong absorption around 1\,\micron\ is a distinctive attribute of He-shell DDet SNe and that this feature can be used to identify and select events relative to normal SNe\,Ia.

\subsection{The Host Environment of He-Shell DDet SNe} \label{sec:host}
\begin{figure*}
    \centering
    \includegraphics[width=\textwidth]{DEIMOS_20jgb.pdf}
    \caption{The SED of the star-forming dwarf galaxy PSO J175312.663+005122.078 (the host galaxy of \sn) and the model from \texttt{prospector}. When fitting the SED with \texttt{prospector}, the DEIMOS spectrum is automatically rescaled to fit the archival photometry from Pan-STARRS \citep[][$g$, $r$, $i$, $z$, $y$ Kron magnitudes]{PS1_2016} and VHS \citep[][$J$ and $K_s$ Petrosian magnitudes]{VHS_2013}. {\it Left:} the SED in the optical band (4750--8350\,\AA\ in the rest frame of the host galaxy). The black line corresponds to the observed spectrum, binned with a size of 2\,\AA. The orange line is the \texttt{prospector} model produced from the median of the stellar population property posterior distributions. The blue shaded region is masked in the fitting owing to the strong telluric lines. {\it Inner panel:} the same comparison, but covering the $g$ through $K_s$ bands (4000--24,000\,\AA). Apart from the spectra, we also show the multiband photometry (green circles) and the best-fit magnitudes (orange squares). {\it Right:} spectra around the most prominent emission lines. {\it Top right:} H$\alpha$, [\ion{N}{2}] $\lambda\lambda$6548, 6583, [\ion{S}{2}] $\lambda\lambda$6716, 6731. {\it Bottom right:} H$\beta$, [\ion{O}{3}] $\lambda\lambda$4959, 5007.}
    \label{fig:host_spec}
\end{figure*}

\begin{figure}
    \centering
    \includegraphics[width=\linewidth]{host.pdf}
    \caption{The sSFR and stellar mass for the host galaxies of He-shell DDet candidates, showing that He-shell DDet SNe can emerge in both star-forming and passive galaxies. The properties for the hosts of SN\,2016hnk and SN\,2018aoz are taken from \citet{Dong_Ca-rich_2022} and the CLU catalog \citep{Cook_2019, de_Ca_rich_2020}, respectively. The gray contours correspond to the bivariate distributions of stellar mass and sSFR for galaxies in the SDSS MPA-JHU DR8 catalog \citep{Kauffmann_SDSS_2003,Brinchmann_SDSS_2004}, visualized using kernel density estimation (KDE) with the data visualization library \texttt{seaborn} \citep{Waskom_seaborn_2021}. Galaxies with BPT classification as AGNs or LINERs are excluded, since certain spectral features (e.g., H$\alpha$ emission) due to nuclear activity might be misinterpreted as being caused by star formation.}
    \label{fig:host}
\end{figure}

We model the host galaxy of \sn\ using \texttt{prospector} \citep{Johnson_prospector_2021}, a package for principled inference of stellar population properties using photometric and/or spectroscopic data. \texttt{Prospector} applies a nested sampling fitting routine through \texttt{dynesty} \citep{Speagle_dynesty_2020} to the observed data and produces posterior distributions of the stellar population properties and model spectral energy distributions (SEDs) with use of \texttt{Python-FSPS} \citep{Conroy_2009,Conroy_2010}. Our observed data include the Galactic-extinction-corrected DEIMOS spectrum, as well as the archival photometric data from the Panoramic Survey Telescope and Rapid Response System \citep[Pan-STARRS;][$g$, $r$, $i$, $z$, $y$ Kron magnitudes]{PS1_2016}  and the VISTA Hemisphere Survey \citep[VHS;][$J$ and $K_s$ Petrosian magnitudes]{VHS_2013}. We use a parametric delayed-$\tau$ star-formation history, given by Equation~(1) of \citet{Nugent_2020} and defined by the $e$-folding factor $\tau$, the Galactic dust extinction law \citep{Cardelli_1989}, and the Chabrier initial mass function \citep{Chabrier_2003} to the model. We further apply a mass-metallicity relation \citep{Gallazzi_2005} to sample realistic stellar masses and metallicities and a dust law that ensures young stellar light attenuates dust twice the amount of old stellar light, as has been observed.  We also add a nebular emission model \citep{Byler_2017} with a gas-phase metallicity and a gas ionization parameter to correctly measure the strength of the emission lines in the DEIMOS spectrum. The model spectral continuum is built from a tenth-order Chebyshev polynomial. We determine the stellar mass and star-formation rate (SFR) from the \texttt{prospector} output, as shown by \citet{Nugent_2022}. The estimated stellar mass is $\log (M_*\,[\Msun])=7.79_{-0.06}^{+0.07}$, and the specific star-formation rate (sSFR) is $\log (\mathrm{sSFR}\,[\mathrm{yr}^{-1}])=-10.25_{-0.08}^{+0.09}$, with the uncertainties denoting the 68\% highest posterior density regions.
  
In Figure~\ref{fig:host}, we show the sSFR and the stellar mass for the host galaxies of six He-shell DDet candidates. Again using \texttt{prospector}, we fit the stellar properties for all the other candidates with optical spectra from the Sloan Digital Sky Survey \citep[SDSS;][]{York_2000} and photometry from the DESI Legacy Imaging Surveys \citep[][$g$, $r$, $z$, $W_1$, $W_2$, $W_3$, $W_4$ magnitudes]{Dey_2019}. With mid-infrared (MIR) photometry available, \texttt{prospector} can better estimate the overall dust extinction in the host galaxy and the contribution of an active galactic nucleus (AGN) to the SED. We therefore add two additional parameters to our \texttt{prospector} fit to sample the MIR optical depth and fraction of AGN luminosity. 

Unfortunately, two hosts (those of the two Ca-Ia objects SN\,2016hnk and SN\,2019ofm) are nearby ($z \lesssim 0.03$) late-type galaxies with extended, spatially resolved spiral structures. Examination of the photometry model from Legacy Surveys (LS) shows that the galaxy aperture does not include the blue, diffuse star-forming regions of these galaxies. Fitting the SDSS spectra + LS photometry would inevitably underestimate their sSFR. For the host of SN\,2016hnk, we instead adopt the results of \citet{Dong_Ca-rich_2022}, which are based on broadband far-ultraviolet (FUV) to far-infrared (FIR) photometry from the $z=0$ Multiwavelength Galaxy Synthesis I \citep[z0MGS;][]{Leroy_2019} to characterize the stellar population with \texttt{prospector}. The SFR they estimated is 1.1\,dex higher than ours, suggesting intense star-formation in the spiral arms.
For the host of SN\,2019ofm, there are no archival stellar population data available; so we redo the photometry using science-ready coadded images from the \textit{Galaxy Evolution Explorer} (GALEX) general release 6/7 \citep[][$FUV$ and $NUV$ bands]{Martin2005a}, the Sloan Digital Sky Survey DR 9 (SDSS; \citealp{Ahn2012a}, $u$, $g$, $r$, $i$, $z$ bands), the Two Micron All Sky Survey \citep[2MASS;][$H$ and $J$ bands]{Skrutskie2006a}, and preprocessed WISE images \citep{Wright2010a} from the unWISE archive \citep[][$W_1$ and $W_2$ bands]{Lang2014a}\footnote{The unWISE images are based on the public WISE data and include images from the ongoing NEOWISE-Reactivation mission R3 \citep{Mainzer2014a, Meisner2017a}, available on \href{http://unwise.me}{http://unwise.me}.}. We use the software package \texttt{LAMBDAR} (Lambda Adaptive Multi-Band Deblending Algorithm in R) \citep{Wright2016a} and tools presented in \citet{Schulze2021a}, to measure the total brightness of the host galaxy. But with the \texttt{LAMBDAR} photometry, the estimated SFR is essentially the same as in the previous fit, suggesting that there is not much ongoing star-formation in the spiral arms. This, along with its moderate sSFR ($\log (\mathrm{sSFR}\,[\mathrm{yr}^{-1}])=-11.27$), indicates the host galaxy is in the transitional phase.

In addition, the host of the normal SN Ia SN\,2018aoz (NGC\,3923) is a local ($z=0.00580$) early-type galaxy and is outside the SDSS footprint, so we adopt its stellar population properties from the Census of the Local Universe (CLU) catalog \citep{Cook_2019, de_Ca_rich_2020}. The other Ca-Ia object PTF\,09dav is not displayed in Figure~\ref{fig:host} because it appears to be hostless, with the nearest galaxy with a known redshift being a star-forming late-type galaxy $\sim$40\,kpc away \citep{Sullivan_2011}. Nonetheless, it is close to several extended sources with low surface brightness, which could be faint dwarf galaxies \citep[see Figure~3 in][]{Kasliwal_2012}. Its nebular-phase spectrum exhibits H$\alpha$ emission, which indicates potential star formation, but could also be explained with photoionized gas around the transient \citep{Kasliwal_2012}.

Figure~\ref{fig:host} reveals that He-shell DDet SNe emerge in both star-forming and passive galaxies.%, which is true for both thin He-shell objects of normal luminosity (SN\,2016jhr in a star-forming host; SN\,2018aoz in a passive host) and thick He-shell, subluminous objects (SN\,2020jgb in a star-forming host; SN\,2018byg in a passive host). 
Their locations in host galaxies also show large variety. SN\,2020jgb has a small projected physical offset ($\sim$0.2\,kpc) from the center of its host, a star-forming dwarf galaxy, so it is likely to originate from a young, star-forming environment. SN\,2016hnk has a moderate projected host offset ($\sim$4\,kpc) and a potential origin in an \ion{H}{2} region with ongoing star formation \citep{galbany_16hnk_2019}. SN\,2019ofm has a large projected offset ($\sim$11\,kpc) but is still on the spiral arm, as shown in its DECaLS image \citep{Dey_2019}. Other objects, including the recently reported SN\,2016dsg and OGLE-2013-SN-079 \citep{Dong_16dsg_2022}, show large projected host offsets ($\gtrsim$10\,kpc) and lie in the galaxy outskirts, which usually indicates origin in an old stellar population.

In this sense, the He-shell DDet sample resembles the normal SN\,Ia population, which can occur in both star-forming and quenched galaxies \citep[e.g.,][]{Sullivan_2006, Smith_2012}. This is very different from some other types of thermonuclear SNe such as Type-Iax SNe (SNe\,Iax), which almost only appear in star-forming galaxies, or 91bg-like and SN\,2002es-like \citep[02es-like;][]{Ganeshalingam_2012} objects, which prefer old stellar environments \citep[see the review by][]{Jha_2019}. This favors the postulated sequence that He-shell DDet SNe may make up a substantial fraction of normal SNe\,Ia, and is supported by stellar-metallicity observations \citep{Sanders_2021, Eitner_2022}.

The diversities in host environments indicate multiple formation channels in the He-shell DDet SN population. Those in star-forming galaxies, \sn\ being the most unambiguous example, could originate from some analogues of the two subdwarf B binaries with WD companions \citep{Geier_2013, Kupfer_2022} discovered in young stellar populations.
On the other hand, those with large host offsets could not be easily formed {\it in situ}. Similarly, many Ca-rich transients are also observed in remote locations \citep[e.g.,][]{Lunnan_2017}, for which some dynamical formation channels have been proposed \citep{Lyman_2014}. To reach the outskirts of galaxies, WD binaries would need to be ejected by globular clusters \citep{Shen_2019} or supermassive black holes \citep{Foley_2015} before explosions. Given that some Ca-rich transients show characteristic DDet properties \citep{de_Ca_rich_2020}, these channels may also be applicable to some of the He-shell DDet SNe. 

The robust detection of \sn\ in a star-forming region also agrees with independent studies of SN\,Ia progenitors using stellar-metallicity observations. After measuring the manganese abundance in the Sculptor dwarf spheroidal galaxy, it is argued in \citet{de_los_reyes_manganese_2020} that sub-\Mch\ SNe\,Ia dominate the chemical evolution of a galaxy, while near-\Mch\ SNe tend to take over at later times. This indicates that observationally, sub-\Mch\ SNe\,Ia might have a stronger preference toward younger stellar populations than near-\Mch\ SNe\,Ia.
Since He-shell DDet is one of the most favored channels to ignite a sub-\Mch\ WD, we expect that the majority of SNe\,Ia in star-forming galaxies (at least the dwarfs) would undergo He-shell DDet. We note that while \sn\ is the first confirmed subluminous He-shell DDet SN in a star-forming dwarf, which indicates that thick He-shell DDet SNe might be intrinsically rare, the same may not be true for those ignited by a thin He-shell since they would look just as  ``normal'' a few days after explosion \citep{Magee_2021}. Unfortunately, few normal SNe\,Ia have been observed at such an early phase to date; thus, we might have missed a great number of thin He-shell DDet SNe. A systematic study based on prompt follow-up observations of infant SNe\,Ia will help verify this implication, with more efficient time-domain surveys in the future.

\section{Conclusions} \label{sec:conclusion}
We have presented observations of \sn, a peculiar SN\,Ia. It has a low luminosity, red $g_\mathrm{ZTF}-r_\mathrm{ZTF}$ colors, and strong line-blanketing in the optical spectra near maximum light, all of which are highly similar to those of SN\,2018byg \citep{de_18byg_2019}, whose observational properties could be explained by the detonation of a shell of helium on a sub-\Mch WD. Fitting the light curves of \sn\ to a grid of models from \citet{polin_observational_2019}, we show that a $\sim$0.87\,$\Msun$ WD beneath a $\sim$0.08\,$\Msun$ He-shell provides a reasonable match to the observed spectrophotometric evolution of \sn. The uncertainty in the host galaxy extinction limits the precision on estimating total progenitor mass, with a reasonable upper limit being $\sim$1.00\,$\Msun$.

A high-SNR NIR spectrum obtained three weeks after maximum light exhibits a prominent absorption feature near 1\,\micron, which could be produced by the unburnt helium (\ion{He}{1} $\lambda$10830) in the outermost ejecta expanding at a high velocity ($\sim$26,000\,\kms). At the same epoch, the \ion{Ca}{2} IRT also has similarly high velocities ($\sim$24,000\,\kms). To date, only a handful of candidate He-shell DDet SNe have observed NIR spectra. Interestingly, all of them show deep absorption features near 1\,\micron, which, if assumed to be \ion{He}{1} $\lambda$10830, would be expanding at a very similar velocity to the HVFs of \ion{Ca}{2} IRT. For these candidates the \ion{Ca}{2} HVFs and putative \ion{He}{1} velocities show significant diversity ranging from $\sim$15,000\,\kms\ in SN\,2016dsg to $\sim$24,000\,\kms\ in \sn. If it is the unburnt helium and the newly synthesized calcium from the He-shell that produce these line features, such a consistency in the expansion rates of different absorption lines would be naturally explained. However, we could not find unambiguous evidence for other \ion{He}{1} absorption lines, such as \ion{He}{1} $\lambda$20581, so we cannot claim a definitive detection of helium in \sn. Nonetheless, alternative possibilities (\ion{Mg}{2}, \ion{C}{1}, \ion{Fe}{2}) that may cause the 1\,\micron\ feature are deemed even less likely. Helium is thus the most plausible explanation for the apparently ubiquitous 1\,\micron\ features.

We propose that He-shell DDet SNe can be robustly identified with NIR spectra. For transients showing a clear 1\,\micron\ feature, to test its potential association with \ion{He}{1} $\lambda$10830 one could follow the checklist below.
\begin{itemize}
    \item Search for \ion{He}{1} $\lambda$20581. A caveat is that one should not always expect to see significant \ion{He}{1} $\lambda$20581 absorption in He-shell DDet SNe, since this line is weaker than \ion{He}{1} $\lambda$10830 and could be almost invisible when the He-shell is thin \citep{Boyle2017_Helium}. The strong telluric lines near 2\,\micron\ also add to the difficulty in detecting \ion{He}{1} $\lambda$20581.
    \item Calculate the line velocity assuming an origin in \ion{He}{1} $\lambda$10830 and check if it is comparable with the HVFs velocity in the \ion{Ca}{2} IRT absorption at a similar phase. While both the detonation recipe in a He-shell DDet model and the viewing angles would affect the observed \ion{He}{1}/\ion{Ca}{2} velocity, we still expect the elements along the line of sight to expand at a similar velocity, if they all have a He-shell origin.
    \item Exclude the possibility of other strong lines. If the NIR spectrum is obtained before the peak brightness of the SN, strong \ion{Mg}{2} and \ion{C}{1} absorption \citep{Hsiao_CSP_2019} would be possible contaminants. Otherwise, if the 1\,\micron\ feature is seen in the transitional-phase spectrum when the inner region of the SN becomes visible, we need to carefully rule out the possibility of an \ion{Fe}{2} origin \citep{Marion2009_NIR}.
\end{itemize}

The He-shell DDet SNe in the tiny sample show diversities in various observational properties, including the peak luminosity, color evolution, chemical abundances, and line velocities, which could be explained by a large variety of He-shell and WD masses \citep{polin_observational_2019,Shen_2D_2021}, viewing angles \citep{Shen_2D_2021}, and the initial chemical compositions in the He-shell \citep{Kromer_DD_2010}. In addition, they are discovered in both old and young stellar populations, \sn\ being the first unambiguous subluminous, thick He-shell DDet candidate in a star-forming galaxy. If, as has been argued \citep[e.g.,][]{Sanders_2021, Eitner_2022}, a substantial fraction of normal SNe\,Ia are triggered by He-shell DDet, then we would naturally expect He-shell DDet SNe to emerge in both star-forming and passive galaxies as normal SNe\,Ia do \citep[e.g.,][]{Sullivan_2006,Smith_2012}, which is exactly what we observe. This is unlike some other subtypes of SNe\,Ia \citep{Jha_2019} which strongly prefer either star-forming galaxies (e.g., SNe\,Iax) or passive galaxies (e.g., 91bg-like and 02es-like objects). Nonetheless, it remains to be examined whether thick He-shell DDet SNe stem from similar progenitors as the majority of normal SNe\,Ia, or if their massive He-shells could only be developed in a completely distinctive population of binary systems.\\

%\begin{acknowledgements}
\noindent We thank Eddie Schlafly and Dustin Lang for suggesting photometry from DESI Legacy Imaging Surveys in SED fitting. We are grateful to Aishwarya Dahiwale, Jillian Rastinejad, and Yuhan Yao for the high-quality spectra they obtained. We also appreciate the excellent assistance of the staffs of the various observatories where data were obtained. K.D. acknowledges support from NASA through the NASA Hubble Fellowship grant \#HST-HF2-51477.001 awarded by the Space Telescope Science Institute, which is operated by the Association of Universities for Research in Astronomy, Inc., for NASA, under contract NAS5-26555. A.V.F. is grateful for financial support from the Christopher R. Redlich Fund and many other individual donors. K.M. is funded by the EU H2020 ERC grant No. 758638. S.S. acknowledges support from the G.R.E.A.T research environment, funded by {\em Vetenskapsr\aa det}, the Swedish Research Council, project number 2016-06012. The Keck II time was provided by CIERA/Northwestern.

This work is based on observations obtained with the Samuel Oschin Telescope 48-inch and the 60-inch Telescope at the Palomar Observatory as part of the Zwicky Transient Facility project. ZTF is supported by the National Science Foundation under Grant No. AST-1440341 and a collaboration including Caltech, IPAC, the Weizmann Institute of Science, the Oskar Klein Center at Stockholm University, the University of Maryland, the University of Washington, Deutsches Elektronen-Synchrotron and Humboldt University, Los Alamos National Laboratories, the TANGO Consortium of Taiwan, the University of Wisconsin at Milwaukee, and Lawrence Berkeley National Laboratories. Operations are conducted by COO, IPAC, and UW. 

SED Machine is based upon work supported by the National Science Foundation under Grant No. 1106171.

This work is also based on observations made with the Nordic Optical Telescope, owned in collaboration by the University of Turku and Aarhus University, and operated jointly by Aarhus University, the University of Turku and the University of Oslo, representing Denmark, Finland and Norway, the University of Iceland and Stockholm University at the Observatorio del Roque de los Muchachos, La Palma, Spain, of the Instituto de Astrofisica de Canarias.

A major upgrade of the Kast spectrograph on the Shane 3\,m telescope at Lick Observatory, led by Brad Holden, was made possible through gifts from the Heising-Simons Foundation, William and Marina Kast, and the University of California Observatories. Research at Lick Observatory is partially supported by a generous gift from Google. The W. M. Keck Observatory is operated as a scientific partnership among the California Institute of Technology, the University of California and NASA; the observatory was made possible by the generous financial support of the W. M. Keck Foundation.

This work was also supported by the GROWTH project \citep{Kasliwal2019} funded by the National Science Foundation under Grant No 1545949.
%\end{acknowledgements}

\facility{PO:1.2m (ZTF), PO:1.5m (SEDM), Gemini:Gillett (GNIRS), Hale (DBSP), NOT (ALFOSC), Shane (Kast Double spectrograph), Keck:I (LRIS), Keck:II (DEIMOS).}
\software{\texttt{astropy} \citep{Astropy_2013, Astropy_2018}, \texttt{CASTRO} \citep{Almgren_Castro_2010}, \texttt{dynesty} \citep{Speagle_dynesty_2020}, \texttt{emcee} \citep{emcee_2013}, \texttt{LAMBDAR} \citep{Wright2016a}, \texttt{matplotlib} \citep{Matplotlib_2007}, \texttt{prospector} \citep{Johnson_prospector_2021}, \texttt{PypeIt} \citep{pypeit:zenodo}, \texttt{pysedm} \citep{Rigault_pysedm_2019}, \texttt{Python-FSPS} \citep{Conroy_2009,Conroy_2010}, \texttt{scipy} \citep{Scipy_2020}, \texttt{seaborn} \citep{Waskom_seaborn_2021}, \texttt{SEDONA} \citep{Kasen_Sedona_2006}.}

\appendix
\begin{figure*}
    \centering
    \includegraphics[width=\textwidth]{model_0_95.pdf}
    \caption{Similar to Figure~\ref{fig:model}, but more models with a total mass of $\sim$0.95\,$\Msun$ and various shell masses (from $0.02\,\Msun$ to $0.13\,\Msun$) are displayed. For these models, we assume no host extinction.}
    \label{fig:model_0_95}
\end{figure*}

\begin{figure*}
    \centering
    \includegraphics[width=\textwidth]{model_1_0.pdf}
    \caption{Similar to Figure~\ref{fig:model}, but more models with a total mass of $\sim$1.00\,$\Msun$ and various shell masses (from $0.02\,\Msun$ to $0.13\,\Msun$) are displayed. For these models, we assume $E(B-V)_\mathrm{host}=0.13$.}
    \label{fig:model_1_0}
\end{figure*}

\section{Comparison to DDet Models with Various Shell Masses}\label{app1}
We have shown that the $r_\mathrm{ZTF}$-band light curve and the observed spectra of \sn\ near peak light are fairly consistent with the DDet of a sub-\Mch\ WD (0.95--1.00\,$\Msun$) beneath a massive shell ($\sim$0.13\,$\Msun$). In the appendix we show the comparison to a variety of DDet models in \citet{polin_observational_2019} with difference shell masses, aiming to discuss how well we can constrain the shell mass of the system.

Figure~\ref{fig:model_0_95} shows the comparison to a few models with a total mass of $\sim$0.95\,$\Msun$, all of which reproduce the brightness of \sn\ in $r_\mathrm{ZTF}$ if there is no host extinction. The $g_\mathrm{ZTF}$-band synthetic light curves, which are closely related to the strength of line-blanketing, are much more heterogeneous. In the two models with thinner shells ($\lesssim$0.08\,$\Msun$), the suppress in flux blueward to $\sim$5000\,\r{A} is much less significant than that seen in \sn\ at $-$4\,days. As a result, they overestimate the brightness in $g_\mathrm{ZTF}$ before peak light. It turns out that for a $0.95\,\Msun$-progenitor, $0.08\,\Msun$ is close to the mass threshold of shell \Ni\ synthesis, above which a significant amount of \Ni\ will be produced in the outermost ejecta, while in thinner shells the radioactive yields are mainly lighter elements such as $^{48}\mathrm{Cr}$ and $^{52}\mathrm{Fe}$. A rich amount of \Ni\ leads to much stronger line-blanketing even before peak light, as is seen in \sn. For this reason, the shell atop the progenitor of \sn\ needs to be well above this threshold. The $0.84\,\Msun+0.11\,\Msun$ model shows the strongest suppression in $g_\mathrm{ZTF}$ among all. For models with even thicker shells, the decay of the radioactive species in the outermost ejecta provide stronger heating, which potentially account for the increase in the brightness in $g_\mathrm{ZTF}$ despite of the level of line-blanketing which also increases.

Figure~\ref{fig:model_1_0} shows models with a total mass of $\sim$1.00\,$\Msun$ assuming $E(B-V)_\mathrm{host}=0.13$. Again all the models reproduce the brightness in $r_\mathrm{ZTF}$. The model with the thinnest shell significantly underestimate the level of line-blanketing, so can be easily ruled out. Models with shells $\gtrsim$0.05\,$\Msun$ behave rather similarly, making the optimal shell mass quite uncertain. Nonetheless, models with shells $\gtrsim$0.10\,$\Msun$ better reproduce the velocity of the HVFs of \ion{Ca}{2} IRT before peak light, which is underestimated in the $0.95\,\Msun+0.05\,\Msun$. We note that all the 1.00\,$\Msun$-models overestimate the brightness of \sn\ in $r_\mathrm{ZTF}$.

We conclude that although none of our models provide a perfect fit to the data, the observations of \sn\ are definitely more consistent with thick-shell models (shell masses $\gtrsim$0.1\,$\Msun$).

\bibliography{SN2020jgb, ZTF, software}
\bibliographystyle{aasjournal}

%% This command is needed to show the entire author+affiliation list when
%% the collaboration and author truncation commands are used.  It has to
%% go at the end of the manuscript.
%\allauthors

%% Include this line if you are using the \added, \replaced, \deleted
%% commands to see a summary list of all changes at the end of the article.
%\listofchanges

%\end{CJK*}
\end{document}

% End of file `sample631.tex'.
