%% Beginning of file 'SN\,2020jgb.tex'
%% using aastex version 6.3
\documentclass[twocolumn]{aastex631}

\newcommand{\sn}{SN\,2020jgb}
\newcommand{\trmax}{$t_{r_\mathrm{ZTF},\mathrm{max}}$}
\newcommand{\tfl}{$t_\mathrm{fl}$}
\newcommand{\Mch}{$M_\mathrm{Ch}$}
\newcommand{\adam}[1]{\textcolor{red}{[AAM: #1]}}
\newcommand{\chang}[1]{\textcolor{blue}{[Chang: #1]}}

\shorttitle{\sn}
\shortauthors{Authors et al.}
\graphicspath{{./}{figures/}}

\begin{document}

\title{\sn}

\author{Authors}
\affiliation{Center for Interdisciplinary Exploration and Research in Astrophysics (CIERA), Department of Physics and Astronomy, Northwestern University, 1800 Sherman Road, Evanston, IL 60201, USA}

\begin{abstract}

\end{abstract}

\keywords{keywords}

\section{Introduction} \label{sec:intro}
It has been clear for decades that Type Ia Supernovae (SNe Ia) are caused by the thermonuclear explosions in carbon-oxygen (C/O) white dwarfs (WDs) in binary systems \citep[see][for a review]{Maoz_2014}. Nevertheless, the nature of the binary companion, as well as how it ignites the WD, remains highly uncertain. 

\citep{polin_observational_2019}, historical papers by Woosley, \citep{Boyle2017_Helium}

The He-shell double detonation (DDet) scenario, in which the detonation of a helium shell atop a WD sends a shock wave inside, such that the runaway thermonuclear synthesis is triggered in the C/O core and inevitably destroy the whole WD, is one of the most promising channels to produce SNe Ia (\textbf{references}). In this scenario, the WD does not have to reach the Chandrasekhar mass (\Mch) to be ignited (\textbf{references}), with a lower limit of ignition mass of $\sim$0.8\,$\mathrm{M_\odot}$ (\textbf{references}). \textbf{Talk about the benchmarks: early red flash caused by the detonation, red color caused by strong Fe-group line blanketing in some certain phase (depending on shell mass), unburnt helium, possible [\ion{Ca}{2}] emission lines in the nebular phase (also mention Ca-rich transients)}.

Using different combos of He-shell mass and C/O core mass, one can reproduce a variety of observables in `normal' SNe Ia with typical luminosities and spectral features near peak light (\textbf{references}), or peculiar sub-luminous ones (\textbf{references}). 

For the DDet SNe that show all the `normal' features near their peaks, the mass of the He-shell is expected to be low \citep[$\lesssim0.01\,\mathrm{M_\odot}$;][]{Kromer_DD_2010,Sim_2010,Shen_DD_2018,polin_observational_2019}. The first candidate of such a thin He-shell SN is SN\,2016jhr \citep{jiang_16jhr_2017}, which exhibits an early red flash and keeps a red $g-r$ color throughout its evolution, though it show a typical $g$-band magnitude at peak for SNe Ia ($M_g\approx-19$). The multi-band light curves involving the early flash and the major peak, as well as the optical spectrum close to the peak light, could be simultaneously fit by a near-\Mch DDet model (a 1.38\,$\mathrm{M_\odot}$ C/O core and a 0.03\,$\mathrm{M_\odot}$ He-shell). Recently, a thinner He-shell is discovered in SN\,2018aoz \citep{Ni_2022}, a SN Ia showing a rapid redward color evolution within $\approx$12\,hr after the first light, which could be explained by a sub-\Mch DDet model (a 1.05\,$\mathrm{M_\odot}$ C/O core and a 0.01\,$\mathrm{M_\odot}$ He-shell). Starting from the third day since the first light, the photometry evolution of SN\,2018aoz is consistent with normal SNe Ia, after the ashes of the thin He-shell becomes optically thin. \textbf{Discuss the importance and difficult of detecting infant SNe Ia.}

In contrast, if the He-shell mass is much greater than 0.01\,$\,\mathrm{M_\odot}$, the ashes of the He-shell detonation could remain optically thick for a long time, hence these thick He-shell DDet SNe are usually red and sub-luminous.

\begin{itemize}
    \item Thick shell
    \begin{itemize}
        \item SN\,2018byg \citep{de_18byg_2019}, red, slow evolution (0.76\,$\mathrm{M_\odot}$+0.15\,$\mathrm{M_\odot}$)
        \item During the late stages of preparing for this paper, \citet{Dong_16dsg_2022} propose another thick shell He-shell DDet candidates, SN\,2016dsg. It exhibits similar spectroscopic features as the archival He-shell single/double detonation candidate OGLE-2013-SN-079 \citep{Dessart_2015, Inserra_OGLE13_079_2015}. \citet{Dong_16dsg_2022} also report their tentative detection of unburnt helium in the NIR spectrum. We will discuss this object in Section~\ref{sec:discussion}.
    \end{itemize}
    \item Ambiguous - strong [\ion{Ca}{2}] emission in the nebular phase and prominent \ion{Si}{2} features, thus known as Ca-rich Ia, also predicted by \citet{polin_nebular_2021}, sub-luminous, line-blanketing, but also some 91bg-like features such as strong \ion{O}{1} absorption, which is not produced by the detonation of the shell
    \begin{itemize}
        \item SN\,2016hnk as a sub-\Mch\ He-shell DDet \citep{jacobson-galan_16hnk_2020} (0.85\,$\mathrm{M_\odot}$+0.02\,$\mathrm{M_\odot}$) or a near-\Mch, 91bg-like object \citep{galbany_16hnk_2019}
        \item SN\,2019ofm \citep{de_Ca_rich_2020}, similar to SN\,2016hnk
    \end{itemize}
\end{itemize}
\section{Observations} \label{sec:obs}
\subsection{Discovery}

%\chang{I do not have access to your .bib file, so my addition of references will be a little mixed and matched.}

\sn\ was first discovered by the Zwicky Transient Facility \citep[ZTF;][]{ZTF2019a,ZTF2019b} on 2020 May 03.463 UT (MJD 58972.463) with the 48-inch Samuel Oschin Telescope (P48) at Palomar Observatory. The automated ZTF discovery pipeline \citep{Masci_2019} detected \sn\ using the image-differencing technique of \citet{Zackay_imagesub_2016}. The candidate passed internal thresholds \citep[e.g.,][]{Mahabal_ZTFML_2019, Duev_ZTFML_2019}, leading to the production and dissemination of a real-time alert \citep{Patterson_ZTFalert_2019} and the internal designation ZTF20aayhacx. It was detected with \chang{$g_\mathrm{ZTF} = 19.86 \pm nn.nn\,$mag} at $\alpha_\mathrm{J2000}=17^\mathrm{h}53^\mathrm{m}12^\mathrm{s}.651$, $\delt_\mathrm{J2000}a=-00^\circ51'21''.81$ and announced to the public in \citet{Fremling_report_2020}. The host galaxy, PSO J175312.663+005122.078, is a dwarf galaxy, to which \sn\ has a projected offset of only $0.3''$. The last non-detection limits the brightness to $r_\mathrm{ZTF} > 20.7$\,mag on 2020 April 27.477 (MJD 58966.477; 5.99\,days before the first detection). This transient was identified as a SN\,Ia by \citet{TNS_2020} due to the prominent \ion{Si}{2} absorption features and a lack of any H features in the spectrum obtained on 2020 May 28.468. \chang{These details are not given in that report, I would simply say - classified as a SN\,Ia in ...}

\subsection{Optical Photometry}
\sn\ was monitored in the $g_\mathrm{ZTF}$ and $r_\mathrm{ZTF}$-bands by ZTF as part of its ongoing Northern Sky Survey \citep{ZTF2019a}. %\chang{The g and r bands used by ZTF are non-standard (relative to SDSS). To make this very clear to the reader, I often write $g_\mathrm{ZTF}$ to be explicitly clear, though that is not necessary.} 
We adopt a Galactic extinction of $E(B-V)=0.404\,$mag \citep{Schlafly2011}, and correct all photometry using the \citet{Fitzpatrick1999} extinction model. We assume there is no additional extinction in the host galaxy. This assumption is supported by the lack of \ion{Na}{1} D absorption at the redshift of the host galaxy, though see \citet{Poznanski_2011} for caveats on the use of \ion{Na}{1} D absorption as a proxy for extinction. 

\chang{it is a little strange to discuss the redshift in a subsection about photometry -- wherever this paragraph goes I would start by stating the observed redshift as measured from the host spectrum. Then discuss peculiar velocities to determine mu.}
To get the distance modulus to \sn, we use the 2M++ model \citep{Carrick2015_2M++} to obtain a peculiar velocity towards its host galaxy, PSO\,J175312.663+005122.078, to be $179$\,km\,s$^{-1}$. This, combined with the recession velocity in the frame of the cosmic microwave background\footnote{See \url{https://ned.ipac.caltech.edu/velocity_calculator}.} (CMB) $v_\mathrm{CMB}=9136$\,km\,s$^{-1}$, give a net Hubble recession velocity is 9307\,km\,s$^{-1}$ \adam{$\pm 250 \kms$}. Adopting $H_0=70$\,km\,s$^{-1}$\,Mpc$^{-1}$, $\Omega_M=0.3$, and $\Omega_\Lambda=0.7$, we estimate the luminosity distance to \sn\ to be 136.1\,Mpc, which yields a distance modulus of $\mu=35.66$.

The forced photometry light curves (absolute magnitudes) in $g_\mathrm{ZTF}$- and $r_\mathrm{ZTF}$-bands are shown in Figure~\ref{fig:photometry}, in which we show all the measurements with $\mathrm{SNR}>2$. \chang{Should point to ZTF forced light curve service as a reference here - probably also make sense to cite Miller et al. in prep since I made the light curve}

\begin{figure*}
    \centering
    \includegraphics[width=\textwidth]{photometry.pdf}
    \caption{\textit{Left}: Comparison of the multi-color light curves  of \sn\ with the normal SN\,Ia SN\,2011fe, the thin shell DDet candidate SN\,2016jhr, and the thick shell DDet candidate SN\,2018byg. The upper (lower) panel shows the evolution in $g$-band ($r$-band) absolute magnitudes. \textit{Right}: comparison of $g-r$ color evolution to SN\,2011fe and SN\,2018byg, as well as 62 normal SNe Ia (open circles) with prompt observations within 5\,days of first light by ZTF \citep{Bulla2020}. The shaded region denotes the 1-$\sigma$ credible interval of the color of SN\,2020jgb until $\sim$60\,days after the peak, estimated using Gaussian process. \chang{I think the shaded region in this figure may be confusing to readers, let's talk about this more. Also – ``credible interval'' suggests a Bayesian analysis has been done and I don't think that's the case}}
    \label{fig:photometry}
\end{figure*}

\subsection{Optical Spectroscopy}\label{sec:optical_spec}
We obtained optical spectroscopic follow-up of the object from $\sim$$-10$\,days to $\sim$$+150$\,days relative to the $r_\mathrm{ZTF}$-band peak, using the Spectral Energy Distribution Machine \citep[SEDM;][]{SEDM_2018} on the automated 60\,inch telescope \citep[P60;][]{P60_2006} at Palomar Observatory, the Kast Double Spectrograph \citep{miller1994kast} at the Shane 3\,m Telescope, the Andalucia Faint Object Spectrograph and Camera (ALFOSC)\footnote{\url{https://www.not.iac.es/instruments/alfosc/}} installed at the Nordic Optical Telescope (NOT), the Double Beam Spectrograph (DBSP) on the 200\,inch Hale telescope \citep[P200;][]{P200_1982}, the Low Resolution Imaging Spectrograph (LRIS) on the Keck I telescope \citep{Keck_1995}. Spectra were reduced using standard procedures \citep[e.g.,][]{Matheson_2000}. Details of the spectroscopic observations are listed in Table~\ref{tab:spec}, and the spectral sequence is shown in Figure~\ref{fig:spec_evo}.

On 2022 March 31, two years after the transient faded, we also took a spectrum for its host galaxy using the DEep Imaging Multi-Object Spectrograph (DEIMOS) on the Keck II telescope \citep{DEIMOS_2003}, for a total integration time of 3200\,s. The spectra were reduced with the \texttt{PypeIt} Python package \citep{pypeit:joss_pub}.

\input{./tables/spec.tex}
\begin{figure*}
    \centering
    \includegraphics[width=\textwidth]{optical_spec_evolution.pdf}
    \caption{\textit{Left}: optical spectral sequence of \sn. Rest frame phases (days) relative to the $r_\mathrm{ZTF}$-band peak and instruments used are posted next to each spectrum. The spectra are after Galactic extinction correction are shown in grey. The black lines are binned spectra with a bin size of 10\,\r{A}, except for the SEDM spectra, whose resolution is lower than the bin size. In the last two spectra, we have subtracted the light from the host galaxy. Only regions with SNR $>2.5$ after binning are plotted. 
    \textit{Right}: spectral comparison with SN\,2018byg (sub-luminous He-shell DDet) and SN\,2004da (normal luminosity). Spectra for SN\,2004da and SN\,2018byg are obtained from the WISEReP repository \citep{wiserep_2012}. \chang{should cits the original sources for the spectra. Also - we should mark regions with telluric absorption that is not corrected}} %The \ion{Si}{2} and \ion{Ca}{2} IRT absorption lines are indicated by the vertical dashed lines. The telluric lines are denoted by the earth symbol, ``$\earth$''.
    \label{fig:spec_evo}
\end{figure*}

\subsection{Near-infrared (NIR) Spectroscopy}
We obtained one NIR (0.8-2.5\,\micron) spectrum of \sn\ using the Gemini near-infrared spectrometer \citep[GNIRS;][]{GNIRS1998} on the Gemini North telescope on 2020 June 9 ($\sim$22\,days after $r_\mathrm{ZTF}$-band peak), for an integration time of 2400\,s. The GNIRS spectrum was reduced with \texttt{PypeIt}.

\begin{figure*}
    \centering
    \includegraphics[width=\textwidth]{NIR_spec.pdf}
    \caption{The NIR spectra of \sn\ and two SNe Ia with normal maximum luminosity \citep[SN\,2004ab and SN\,2004da,][]{Marion2009_NIR}, taken about three weeks after the peak. For each spectrum, the continuum at $\gtrsim$1.2\,\micron\ is significantly reshaped by the Fe-group blanketing (emission features, blue vertical lines) and \ion{Co}{2} absorption (pink vertical lines). Spectra for SN\,2004ab and SN\,2004da are obtained from \citet{Marion2009_NIR}.}
    \label{fig:NIR_spec}
\end{figure*}

\section{Analysis} \label{sec:analysis}
\subsection{Photometric Properties}
\sn\ exhibited a fainter light curve than normal SNe Ia. In Figure~\ref{fig:photometry}, we compare the photometric properties of \sn\ with the nearby, well-observed SN\,2011fe \citep{Nugent_11fe_2011} and two He-shell DDet candidates, including the normal-luminosity thin shell candidate SN\,2016jhr \citep{jiang_16jhr_2017} and the sub-luminous thick shell candidate SN\,2018byg \citep{de_18byg_2019}, with available photometric data on the Open Supernova Catalog\footnote{See \url{https://github.com/astrocatalogs/supernovae}.} \citep{Guillochon_2017}. \chang{were all these light curves corrected for reddening? I thought 11fe peaked brighter than $-19$. Also might be worth a small note stating that these were all observed in different g/r bands, which also means it's worth noting (somewhere not necessarily here) that $K$-corrections have not been performed}

While the observational coverage is sparse in the rise to maximum light, from Figure~\ref{fig:photometry} it is clear that \sn\ is less luminous than normal SNe\,Ia (e.g., SN\,2011fe). Furthermore, there is a flatter evolution in the $r_\mathrm{ZTF}$ evolution between $-14$\,d and maximum light for both \sn\ and SN\,2020jgb than there is for SN\,2011fe.  

In the right panel of Figure~\ref{fig:photometry}, we compare the color evolution ($g-r$) of these objects relative to the measured time of first light \tfl, accompanied by 62 normal SNe Ia (open circles) observed within 5 days of \tfl\ by ZTF \citep[from][]{Bulla2020}. For \sn\, the early rise of the light curve was not well sampled, so we estimate \tfl\ as the midpoint of the first detection and the last non-detection. We adopt an uncertainty on this estimate of 3\,days. We also estimate the 1-$\sigma$ uncertainty of the $g_\mathrm{ZTF}-r_\mathrm{ZTF}$ color (the shaded region) in \sn\ by fitting the forced photometry light curves in individual bands with Gaussian process (GP). The GP regression is conducted with the \texttt{gaussian\_process} module in the \texttt{scikit-learn} package \citep{scikit-learn}, in which a combination of the radial-basis function (RBF) kernel and the white kernel is adopted. Measurements with lower SNR ($1<\mathrm{SNR}<2$) are also included in the fit to better model the decline in both bands $\gtrsim$30\,days after the peak. All three DDet candidates are undoubtedly redder than normal SNe Ia. At peak light, \sn\ was not as red as the extreme case, SN\,2018byg ($g_\mathrm{ZTF}-r_\mathrm{ZTF}\approx2.2$\,mag \chang{I think the colors for 18byg are from P60 so they aren't the ZTF filters}), but exhibited a similar color as SN\,2016jhr ($g-r\approx0.5$).

\subsection{Optical Spectral properties}
In Figure~\ref{fig:spec_evo}, we show the optical spectral sequence of \sn, and compare its spectra with some other SNe Ia near peak luminosity. For the spectra obtained after +100\,d there is clear contamination from the host-galaxy, including the presence of narrow emission lines. For these spectra we subtract the galaxy light as measured in the DEIMOS spectrum from 2022 (see Section~\ref{sec:optical_spec}). The earliest spectrum was obtained by SEDM $\approx$10\,days before the $r_\mathrm{ZTF}$-band peak. We only show portions of the spectrum where the $\mathrm{SNR}>2.5$, where the continuum is almost featureless with some marginal detection of the \ion{Si}{2} $\lambda$6355 at $\approx$6100\,\r{A}, the trademark of SNe Ia. In subsequent spectra the \ion{Si}{2} features become more prominent and are clearly detected through +12\,d. We measure \ion{Si}{2} expansion velocities with a Gaussian profile following a similar procedure as in \citet{Maguire_2014}. The fitting region is selected by visual inspection. The continuum is assumed to be linear. Within the model, the continuum flux density at the blue and red edges are free parameters for which we adopt a normal distribution as a prior. The mean and standard deviation for the normal are the observed flux density and it's uncertainty, respectively, at each edge of the fitting region. \chang{Is this done prior to the Gaussian fit? I thought the continuum was estimated at the same time as the Gaussian profile?} The absorption feature is then normalized by the continuum and fit to a single Gaussian profile with three parameters (amplitude, mean velocity, velocity dispersion). \chang{what priors for these parameters?} Posteriors of these five parameters are sampled with \texttt{emcee} \citep{emcee_2013} using the Markov chain Monte Carlo (MCMC) method. We find the mean expansion velocity is $\approx$11,500\,km\,s$^{-1}$ near maximum light.

In many SNe Ia the \ion{Ca}{2} infrared triplet (IRT) absorption has two distinct components. Following, \citet{Maguire_2014}, we refer to these components as photospheric-velocity features (PVFs) and high-velocity features (HVFs) \chang{double check that Kate was first to use this terminology}. The PVFs originate from the main line-forming region with typical photospheric (i.e., bulk ejecta) velocities, while the HVFs are blueshifted to much shorter wavelengths, indicating significantly higher (by greater than $\sim$6000\,km\,s$^{-1}$) velocities than typical PVFs \citep{Silverman_HVF_2015}. Figure~\ref{fig:spec_evo} shows that \sn\ has prominent HVFs of \ion{Ca}{2} IRT. The HVFs are visible in our first spectrum of \sn, and remain prominent through $+36$\,days. Using a similar technique in modeling the \ion{Si}{2} features, we fit the HVFs with multiple Gaussian profiles assuming each line in the triplet can be approximated by the same profile (i.e., same amplitude and velocity dispersion), and obtained a best-fit expansion velocity $\gtrsim$26,000\,km\,s$^{-1}$. A clear delineation between the HVFs and PVFs is visible $\approx$4\,days before peak light. Since then we fit the broad absorption features with two different velocity components simultaneously. The velocity of HVFs slightly declines but stays above $\approx$24,000\,km\,s$^{-1}$, and the velocity of PVFs declines from $\approx$11,000\,km\,s$^{-1}$ to $\approx$9,000\,km\,s$^{-1}$. As in normal SNe Ia, the relative strength between the HVFs and PVFs decreases with time. \chang{might be worth a figure - especially if we have this measured for several other DDet now}

The nebular phase spectra of \sn\ are dominated by the Fe-group elements, showing some enhancement in flux between $\approx$4500 and $\approx$6000\,\r{A}. We did not detect any emission feature related to [\ion{Ca}{2}] $\lambda\lambda$7291, 7324, which is a hallmark for Calcium-rich gap transients and is also prominent in a few DDet candidates \citep[e.g., SN\,2016hnk and SN\,2019ofm;][]{De_Ca-rich_2020}. 

The optical spectral evolution of \sn\ resembles that of SN\,2018byg, a sub-luminous DDet candidate with a potentially thick helium shell. \chang{I think we can safely call this a DDet, and not a candidate} At early times, both SNe were relatively blue and featureless with broad and shallow \ion{Ca}{2} IRT absorption. As they evolved closer to maximum light, they developed strong continuous absorption bluewards of $\approx$5000\,\r{A}, while the \ion{Si}{2} $\lambda$6355 and the \ion{Ca}{2} IRT became more prominent. \ion{S}{2} was not detected in either object. In the DDet scenario, a large amount of Fe-group elements will be synthesized in the outer regions of the ejecta, which will cause significant line-blanketing near maximum light \citep{Kromer_DD_2010, polin_observational_2019}, and also high velocity intermediate-mass elements like \ion{Ca}{2} \citep{Fink_DD_2010, Kromer_DD_2010}. The similarity between \sn\ and SN\,2018byg, makes \sn\ another promising DDet SN candidate. 

SN\,2004da is a normal SN\,Ia that shows similarities to \sn\ in the NIR (Section~\ref{sec:NIR_spec}), however, the two SNe are extremely different in the optical (Figure~\ref{fig:spec_evo}). From this comparison it is clear that \sn\ is not a normal SN\,Ia. 

\subsection{NIR Spectral properties}
\label{sec:NIR_spec}
The NIR spectrum of \sn\ is compared with two normal SNe Ia at a similar phase in Figure~\ref{fig:NIR_spec} \citep[data for SNe\,2004ab and 2004da from][]{Marion2009_NIR}. \sn\ shows a strong absorption feature at $\sim$0.99\,\micron, which is not seen in normal SNe Ia. This feature was still significant two weeks later, as detected by LRIS on Keck (see Figure~\ref{fig:hvf_comp}), though it was only partially covered. Aside from this prominent feature, \sn\ resembles normal SNe Ia in NIR band. The shape of the continuum redwards to $\approx$1.2\,\micron\ is significantly altered by line-blanketing from Fe-group elements. Just like normal SNe Ia, \sn\ shows an enhancement of flux at about 1.3, 1.55, 2.0, 2.1, and 2.25\,\micron, accompanied by several \ion{Co}{2} absorption lines. It is especially similar to SN\,2004da at +25\,days after maximum light as the steep increase in flux at $\approx$1.55\,\micron, known as the \textit{H}-band break \citep{Hsiao_CSP_2019}, has become less prominent.%For many other SNe Ia, however, the \textit H-band break can last till much later stages \citep{Marion2009_NIR}.

In a sample of \chang{NNN} SNe\,Ia with NIR spectra obtained more than 15\,days after peak, not a single one shows prominent absorption features around 1\,\micron \citet{Marion2009_NIR}. We have investigated several potential identifications for this feature (see below), however, none of them provide a fully satisfying explanation.

The most tantalizing possibility is that the absorption is due to \ion{He}{1} $\lambda$1.0830\,\micron \chang{I think (?) the $\lambda$ convention does not specify units because it is assumed to be \AA, in which case this would just be 10,830. Worth double checking}. If \sn\ is a DDet SN, then unburned He could lead to observed absorption in the spectrum, as shown in the sub-Chandrasekhar-mass He-shell DDet models of \citet{Boyle2017_Helium}. Figure~\ref{fig:hvf_comp} shows that the 1\,\micron\ feature, if associated with \ion{He}{1} $\lambda$1.0830\,\micron, has a high velocity ($\sim$26,000\,km\,s$^{-1}$), especially considering the phase of the SN. The \ion{Ca}{2} IRT also exhibits similarly high velocities at the same phase ($\sim$24,000\,km\,s$^{-1}$), suggesting it is not impossible to see very high absorption velocities at this phase. The expansion velocity in the ejecta is roughly linearly proportional to the radius, so such a high velocity indicates that both the \ion{Ca}{2} IRT and the tentative \ion{He}{1} absorption line form far outside the normal photosphere, which has a velocity of only $\approx$10,000\,km\,s$^{-1}$. In this sense, the He-shell DDet scenario, in which the unburnt helium is located in the outermost ejecta, is indeed supported.
\begin{figure*}
    \centering
    \includegraphics[width=\textwidth]{CaII_HeI_hvf.pdf}
    \caption{Spectra of \sn, SN\,2018byg, and SN\,2016hnk in the velocity space, comparing the \ion{Ca}{2} IRT absorption features (upper panels) and the 1\,\micron\ features assuming they are associated with \ion{He}{1} $\lambda$1.0830\,\micron\ (lower panels). The red dashed lines mark the minimum of each 1\,\micron\ feature. Data for SN\,2018byg and SN\,2016hnk are obtained from the WISEReP repository \citep{wiserep_2012}.}
    \label{fig:hvf_comp}
\end{figure*}

We cannot claim an unambiguous detection of \ion{He}{1}, however, as our spectra lack definitive absorption from other He features that we would expect to be prominent, such as \ion{He}{1} $\lambda$2.0581\,\micron. Considering a line velocity of $\approx$26,000\,km\,s$^{-1}$ and a host galaxy redshift of 0.0309, this line will be blueshifted to $\approx$1.95\,\micron\ in the observer frame, so will be strongly blended by the strong telluric lines within 1.8-2.0\,\micron. After telluric correction, the signal to noise ratio reaches $\sim$5, with which we still cannot see any significant absorption feature. An upper limit of the equivalent width is determined to be $<$$2\%$ of the 1.0830\,\micron\ line, while theoretically, the 2.0581\,\micron\ line is supposed to be only a factor of 6-12 weaker, depending on temperature \citep{Marion2009_NIR}. Another fact is that the 1\,\micron\ feature is as strong as the \ion{He}{1} $\lambda$1.0830\,\micron\ in many helium-rich core-collapse supernovae, say, Type Ib supernovae, in which the \ion{He}{1} $\lambda$2.0581\,\micron\ is weaker than the 1.0830\,\micron\ line yet still prominent \citep{CSP_Ibc_2022}. If the 1\,\micron\ feature is associated with \ion{He}{1}, it would be very unusual if the 2\,\micron\ feature is not seen at all, even if somehow blended by the telluric lines. \chang{There should be a note here about the strength of the lines in the Boyle paper... for one of their models there is no obvious 2 micron absorption} \adam{that model is for `normal' SNe\,Ia}

Other possible identifications include \ion{Mg}{2} $\lambda$1.0927\,\micron, \ion{C}{1} $\lambda$1.0693\,\micron, and \ion{Fe}{2} $\lambda$1.0500\,\micron\ \& $\lambda$1.0863\,\micron. The \ion{Mg}{2} $\lambda$1.0927\,\micron\ line is prevalent in the NIR spectra of SNe Ia, but usually disappears within a week after peak \citep{Marion2009_NIR}, while the 1\,\micron\ feature was still visible over a month after the peak in the Keck LRIS spectrum. The required radial velocity is $\approx$30,000\,km\,s$^{-1}$, $\approx$20\% faster than the HVFs of \ion{Ca}{2} IRT at the same phase. While such a high velocity for \ion{Mg}{2} has never been seen in other SNe Ia, since high-velocity intermediate-mass elements like magnesium and calcium can be synthesized by the detonation of helium shell \citep{Shen_DD_2014}, the \ion{Mg}{2} origin of the 1\,\micron\ feature cannot be strictly ruled out. But if we attribute this 1\,\micron\ feature to high-velocity \ion{Mg}{2}, we would expect an even stronger $\lambda$0.9227\,\micron\ line to be blueshifted to the red edge of the \ion{Ca}{2} IRT, which is not detected. Given the strength of the 1\,\micron\ feature, the 0.9227\,\micron\ line should not be completely obscured by the \ion{Ca}{2} IRT features. 

\ion{C}{1} $\lambda$1.0693\,\micron\ is not observed as frequently as \ion{Mg}{2} $\lambda$1.0927\,\micron in SNe\,Ia. \citet{Hsiao_CSP_2019} presented a sample of five SNe Ia with \ion{C}{1} detections, showing the \ion{C}{1} feature is strongest for fainter, fast-declining objects. However, in their sample, the \ion{C}{1} feature is a pre-maximum feature which fades away as the luminosity peaks, so the discrepancy in phase is large. The required expansion velocity $\approx$22,000\,km\,s$^{-1}$, which is overwhelmingly faster than the estimated carbon velocity for the sample in \citet{Hsiao_CSP_2019} ($\sim$10,000-12,000\,km\,s$^{-1}$), but still consistent with the HVFs of \ion{Ca}{2} IRT. Nonetheless, no significant carbon absorption is detected in the optical band.

The \ion{Fe}{2} features in SNe Ia usually start to develop roughly three weeks after the peak, which is about the same phase as we obtained our GNIRS spectrum. Two \ion{Fe}{2} lines, $\lambda$0.9998\,\micron\ and $\lambda$1.0500\,\micron, are actually visible on the blue/red wings of the 1\,\micron\ feature. The \ion{Fe}{2} $\lambda$1.0863\,\micron\ line is not yet seen in the GNIRS spectrum. They correspond to an expansion velocity of $\approx$8,000\,km\,s$^{-1}$, which is consistent with the PVFs of the \ion{Ca}{2} IRT at the same epoch. They also match the same two lines for normal SNe Ia \citep{Marion2009_NIR}, making the identification more reliable. Obviously, these two \ion{Fe}{2} features are wider and shallower than the strong feature between them. We fit the 1\,\micron\ feature with three Gaussian profiles. Two of them are set to be the blueshifted \ion{Fe}{2} $\lambda$0.9998\,\micron\ and $\lambda$1.0500\,\micron, and the other is an uncorrelated Gaussian profile which mainly describes the absorption in the center. We find that the shallower and wider \ion{Fe}{2} lines only make up $\sim$40\% of the total equivalent width, and the rest $\sim$60\% comes from the central feature, which cannot be accounted for by any \ion{Fe}{2} feature at the same velocity. Given the similarity of the Fe-group line-blanketing between the GNIRS spectrum with the spectrum of SN\,2004da at +25\,days, the distribution of Fe-group elements inside each supernova ejecta should be somehow similar as normal SNe Ia, so the central region of the 1\,\micron\ feature is not likely to be associated with \ion{Fe}{2} either.

While the nature of the 1\,\micron\ feature remains uncertain, other He-shell DDet candidates also seem to show similar complexity in this region. In the currently small sample of six candidates, three objects (SN\,2016jhr, SN\,2018aoz, and SN\,2019ofm) do not have any available NIR spectra, while the other three (at quite different phases though) all exhibit strong absorption features near 1\,\micron, as shown in Figure~\ref{fig:NIR_comp}. SN\,2016hnk has two deep absorption features around 1.02\,\micron\ and 1.17\,\micron, both are at a longer wavelength than the 1\micron\ feature in \sn. \citet{galbany_16hnk_2019} suggest both of them could be caused by \ion{Fe}{2}, though they are deeper than in other SNe Ia. The velocity of the 1.02\,\micron\ feature is $\approx$21,000\,km\,s$^{-1}$ assuming a \ion{He}{1} $\lambda$1.0830\,\micron\ origin, which, just like \sn, is about the same as the HVFs of the \ion{Ca}{2} IRT in the optical spectra (see Figure~\ref{fig:hvf_comp}). The PVFs of the \ion{Ca}{2} IRT of both SNe have a similar expansion velocity of $\approx$10,000\,km\,s$^{-1}$. Such a consistency in velocities in also seen in SN\,2018byg (see Figure~\ref{fig:hvf_comp}). But given the exotic width and lower signal-to-noise ratio in the 1\,\micron\ feature, the exact line velocity is hard to determine. It is likely to be a mixture of several different lines. 

\citet{Dong_16dsg_2022} recently proposed another thick He-shell DDet candidate, SN\,2016dsg, with an absorption line around 0.97-1.05\,\micron\ in a low-SNR NIR spectrum at $+16.6$\,days\footnote{SN\,2016dsg was discovered declining. The phase is relative to the discovery time.}. Assuming \ion{He}{1} $\lambda$1.0830\,\micron\ origin, the minimum of the absorption profile (at $\approx$1.03\,\micron, see Figure~4 in \citealp{Dong_16dsg_2022}) corresponds to an expansion velocity of $\approx$15,000\,km\,s$^{-1}$. This is lower than the 1\,\micron\ features in \sn\ and SN\,2016hnk assuming their \ion{He}{1} origin. Interestingly, SN\,2016dsg shows the least prominent HVFs of \ion{Ca}{2} IRT among the four, which also has a low velocity of $\approx$15,000\,km\,s$^{-1}$. Once again, the scenario where both the unburnt helium and the high velocity calcium are located at the outermost shell is favored.

Unfortunately, none of the spectra for SN\,2016dsg, SN\,2016hnk, or SN\,2018byg covers the 2\,\micron\ region, thus it is not possible to identify the presence of helium decisively. But if the 1\,\micron\ feature of the these objects are of the same origin, they are more likely to be correlated with the high velocity ejecta lying in the outmost region in the supernovae, because at least for \sn\ and SN\,2016hnk, the difference in their photospheric velocities cannot explain the discrepancy in their line velocities of the 1\,\micron\ feature. Then helium is still a promising candidate to cause strong absorption near 1\,\micron\ for these sub-luminous He-shell DDet SNe Ia. 

In conclusion, from all the He-shell DDet candidates with NIR spectra available, we have detected strong absorption features near 1\,\micron, which is not seen in normal SNe Ia. Indeed, these candidates have their NIR spectra taken at different epochs, hence each 1\,\micron\ feature can be of completely unrelated origin. Had they all originate from \ion{He}{1} $\lambda$1.0830\,\micron, there would still be large diversity in the corresponding expansion velocities. This is to be confirmed in a more complete NIR spectral sequence in future He-shell DDet SNe Ia. Nonetheless, the seemingly ubiquitous 1\,\micron\ feature in various phases is possibly a distinctive attribute against normal SNe Ia.

\begin{figure}
    \centering
    \includegraphics[width=\linewidth]{NIR_spec_comp.pdf}
    \caption{The NIR spectra (9000 to 13000\,\r{A}) of a few normal SNe Ia (SN\,2011fe and SN\,2004da) and three He-shell DDet candidates, which are all sub-luminous SNe Ia (SN\,2016hnk, SN\,2018byg, and this source, \sn). Other than the spectrum of SN\,2004da, all spectroscopic data are obtained from the WISEReP repository \citep{wiserep_2012}.}
    \label{fig:NIR_comp}
\end{figure}

\section{Host Galaxy} \label{sec:host}
We obtained a DEIMOS spectrum of the host galaxy, PSO J175312.663+005122.078, on 2022 March 31. The host exhibits strong, narrow emission lines including H$\alpha$, H$\beta$, [\ion{N}{2}] $\lambda$6583, [\ion{O}{3}] $\lambda$5007, and [\ion{S}{2}] $\lambda$6716 \& $\lambda$6731. By fitting all these emission features with Gaussian profiles we obtain an average redshift of $z=0.0309\pm0.0003$. With the diagnostic emission line equivalent width ratios ($\log$~[\ion{N}{2}]/H$\alpha=-1.19\pm0.07$ and $\log$~[\ion{O}{3}]/H$\beta=0.53\pm0.06$), the host is consistent with star-forming galaxies in the BPT diagram \citep{BPT_1981, Veilleux_1987}. 

We model the detailed properties of the host galaxy using \texttt{prospector} \citep{Johnson_prospector_2021}, a package for principled inference of stellar population properties using photometric and/or spectroscopic data. The input data included the Galactic extinction corrected DEIMOS spectrum, as well as the archival photometric data from the Panoramic Survey Telescope and Rapid Response System \citep[Pan-STARRS;][{\it r, i, z} Kron magnitudes]{PS1_2016}  and the VISTA Hemisphere Survey \citep[VHS;][J and $\mathrm{K}_\mathrm{s}$ Petrosian magnitudes]{VHS_2013} \chang{didn't we change to LS?}. In the best fit, the estimated stellar mass is $\log (M_*\,[\mathrm{M_\odot}])=7.79_{-0.06}^{+0.07}$, and the specific star-formation rate (sSFR) is $\log (\mathrm{sSFR}\,[\mathrm{yr}^{-1}])=-10.25_{-0.08}^{+0.09}$, with the uncertainties denoting the 68\% credible regions.

In Figure~\ref{fig:host}, we show the sSFR and the stellar mass for the host galaxies of our \chang{these are not all ``ours'' - others discovered several of these candidates} six He-shell DDet candidates. Again using \texttt{prospector}, we fit the stellar properties for all the other candidates with optical spectra from the Sloan Digital Sky Survey \citep[SDSS;][]{York_2000} and photometry from the DESI Legacy Imaging Surveys \citep[][{\it g, r, z, $W_1$, $W_2$, $W_3$, $W_4$} magnitudes]{Dey_2019}. With mid-infrared photometry available, \texttt{prospector} can better estimate the overall dust extinction in the host galaxy and the contribution of an active galactic nucleus (AGN) to the spectral energy distribution (SED). Unfortunately, two (SN\,2016hnk and SN\,2019ofm) out of six hosts are close-by ($z\lesssim 0.03$) late-type galaxies with extended, spatially resolvable spiral structures. In both surveys, only photons from their red, concentrated bulges were fed to the detectors, while the lights from the blue, diffusive star-forming regions were completely missed. We would inevitably underestimate their SFR should we naively fit the SEDs from these surveys. Therefore, for the host of SN\,2016hnk, we adopt the results in \citet{galbany_16hnk_2019} as part of the PMAS/PPak Integral-field Supernova Hosts Compilation \citep[PISCO;][]{Galbany_PISCO_2018}, where the photons from the \ion{H}{2} regions in the spiral arms were also collected using integral field spectroscopy (IFS). For the host of SN\,2019ofm, there are no IFS data available, so we still show our best-fit stellar mass and sSFR in Figure~\ref{fig:host}, with the caveat that the sSFR should be regarded as a lower limit. The host of SN\,2018aoz (NGC\,3923) is a local ($z=0.00580$) early-type galaxy and is outside the SDSS footprint, so we adopt its stellar population properties from the Census of the Local Universe (CLU) catalog \adam{References?}.

The recent study on SN\,2016dsg and OGLE-2013-SN-079 \citep{Dong_16dsg_2022} enlarges the sample of thick He-shell DDet candidates. Similar to SN\,2018byg, both of the them reside in the outskirts of early-type galaxies which are $>$10\,kpc away from the host centers, indicating an origin in old stellar populations. A projected host offset $\gtrsim$10\,kpc is also seen in SN\,2016jhr, SN\,2018aoz, and SN\,2019ofm, whereas the hosts of SN\,2016jhr and SN\,2019ofm are star-forming galaxies. SN\,2016hnk, on the contrary, has a smaller projected host offset of $\sim$4\,kpc and a potential origin in an \ion{H}{2} region with ongoing star-formation \citep{galbany_16hnk_2019}. \sn\ is the first He-shell DDet candidate in a star-forming dwarf galaxy. It has has the smallest projected physical offset ($\sim$0.2\,kpc) yet to know.

Given the limited sample size yet, the host environments of the He-shell DDet candidates have started to show diversity. This is true for both thin He-shell objects of normal luminosity (SN\,2016jhr in a star-forming host; SN\,2018aoz in a quenched host) and thick He-shell, sub-luminous objects (SN\,2020jgb in a star-forming host; SN\,2016dsg, SN\,2018byg, and OGLE-2013-SN-079 in quenched hosts), even when the natures of SN\,2016hnk and SN\,2019ofm remain ambiguous. In this sense, the He-shell DDet sample resembles the SNe Ia population in general, which can occur in both star-forming and quenched galaxies \citep[e.g.,][]{Sullivan_2006, Smith_2012}. But such a diversity in host environments is very different from some other types thermonuclear supernovae, such as Type-Iax supernovae (SNe Iax) which almost only appear in star-forming galaxies, or SN\,1991bg-like and SN\,2002es-like objects, which prefer old stellar environments \citep[see the review in][]{Jha_2019}. Most SNe in our sample show a relative large host galaxy offset, which has also been observed in many Ca-rich transients \citep{Lunnan_2017}, with typical projected offsets being $\sim$10-100\,kpc \citep{de_Ca_rich_2020}. Some dynamical formation channels have been proposed to explain the remote locations of Ca-rich transients, such that WD binaries would need to be hardened and ejected by globular clusters \citep{Shen_2019} or supermassive black holes \citep{Foley_2015} before explosions, which may also be the case for some, if any all, of He-shell DDet SNe. Other DDet SNe, \sn\ being the most unambiguous example, are located in star-forming regions and are thus highly likely to be formed {\it in situ}. Two recently discovered subdwarf B binaries with WD companions which were found in young stellar populations could be promising progenitor systems \citep{Geier_2013, Kupfer_2022}.

The robust detection of He-shell DDet SNe in star-formation regions also agrees with independent studies on SNe Ia progenitors using stellar metallicity observations. \citet{de_los_reyes_manganese_2020}, by studying the manganese abundances in dwarf spheroidal satellites of the Milky Way, argue that sub-\Mch\ SNe Ia dominates the chemical evolution of a galaxy, while near-\Mch\ SNe tend to take over at later times, indicating that observationally, sub-\Mch\ SNe Ia might have a stronger preference towards younger stellar populations than near-\Mch\ SNe Ia. He-shell DDet is one of the most favored channel to ignite a sub-\Mch\ WD, which could produce both normal and sub-luminous SNe Ia. Consequently, we would expect the majority of exploded SNe Ia in star-forming galaxies (at least the dwarfs) would undergo DDet. We note that while \sn\ is the first confirmed He-shell DDet SN in a star-forming dwarf, which indicates that thick shell DDet SNe might be intrinsically rare, the same may not be true for thin shell DDet SNe since they would look just as {\it normal} in both photometric and spectroscopic evolution a few days after explosions \citep{Ni_2022}. Unfortunately, few of them are observed in such an early phase to date (SN\,2016jhr and SN\,2018aoz being two of them), thus we might have missed a great number of He-shell DDet SNe. With more efficient time domain surveys kicking in in the near future and prompt follow-up observations being increasingly available (\textbf{also need some references}), a systematic studies on the infant SNe Ia will help confirm this implication. \adam{Do you think we should put this paragraph here or in the discussion section?}
\begin{figure}
    \centering
    \includegraphics[width=\linewidth]{host.pdf}
    \label{fig:host}
    \caption{The specific star-formation rate (sSFR) and the stellar mass for the host galaxies of He-shell DDet candidates. The properties for the hosts of SN\,2016hnk and SN\,2018aoz are taken from \citet{galbany_16hnk_2019} and the CLU catalog \adam{References?}, respectively. For the sSFR in the host of SN\,2019ofm, only a lower limit is shown (the triangle). The background is a sample of galaxies from the SDSS MPA-JHU DR8 catalog \citep{Kauffmann_SDSS_2003,Brinchmann_SDSS_2004}. Galaxies with BPT classification as AGNs or LINERs are excluded, since certain spectral features (e.g., H$\alpha$ emission) due to nuclear activities might be misinterpreted as star formation.}
\end{figure}

\section{Model Comparisons} \label{sec:model}
{\it A few sentences from Abi describing the models.}

In Figure~\ref{fig:model}, we show the comparison of the photometric and spectroscopic features of \sn\ with DDet models from \citet{polin_observational_2019}. The peak luminosity reflects the total progenitor mass (C/O core $+$ He shell), and we find models with a total mass of $0.95\,\mathrm{M_\odot}$ generally reproduce the $r$-band peak brightness well. Thus in Figure~\ref{fig:model}, the total mass is fixed to be $0.95\,\mathrm{M_\odot}$ in all models. The overall $r$-band photometric evolution is best fit by the model with a $0.87\,\mathrm{M_\odot}$ C/O core and a $0.08\,\mathrm{M_\odot}$ He shell, while all three models underestimate the $g$-band brightness after the peak. This deviation may be attributed to a variety of factors on handling the explosion and radiative transfer. First, throughout the simulations we assume local thermodynamic equilibrium (LTE), which is not valid once the ejecta becomes optically thin. Typically the bulk ejecta of a sub-Chandrasekhar SNe Ia remains optically thick for $\sim$30\,days since the explosion. But in modeling the brightness $g$-band, the LTE assumption is even more tricky, because the major opacity in $g$-band comes from the Fe-group line-blanketing in the outermost ejecta, where the optical depth may evolve differently from that at the photosphere. Hence the LTE condition may quickly become inapplicable. Furthermore, our 1-D He shell model is not capable to capture the multi-dimensional effects in the explosion such as asymmetries. The viewing angle is known to have a significant influence on the observed light curves \citep{Kromer_DD_2010, Sim_2012, Gronow_2020, Shen_2021}, especially in bluer bands where the line-blanketing depends sensitively on the distribution of He-shell ashes \citep{Shen_2021}. In previous studies on other DDet objects, the $g$-band brightness is systematically under-predicted shortly after the peak, despite the fact that redder bands can be fit decently \citep[e.g.][]{jiang_16jhr_2017,jacobson-galan_16hnk_2020}.

The model which best fits the photometry ($0.87\,\mathrm{M_\odot}+0.08\,\mathrm{M_\odot}$) also reproduces the major absorption features (e.g., Fe-group line-blanketing, \ion{Si}{2} $\lambda$6355, PVFs of \ion{Ca}{2} IRT) and the corresponding expansion velocities near the peak light. 
However, we are not able to fit the continua in $g$-, $r$-, and $z$-bands simultaneously, and the strong \ion{Ca}{2} HVFs are not seen in the best-fit spectrum. These discrepancies could also be due to the asymmetry in the DDet, that \sn\ was observed fairly close to the ignition point, where the abundances of Fe-group elements and high velocity calcium synthesized in the shell could be much higher than an angle-averaged 1-D model would predict.
In addition, the predicted \ion{Si}{2} $\lambda$5972 does not show up in the observed spectrum.

{\it Conclusions from the model comparison.}

\begin{figure*}
    \centering
    \includegraphics[width=\textwidth]{model.pdf}
    \label{fig:model}
    \caption{{\it Left:} Comparison of the photometric evolution of \sn\ with the He-shell DDet models from \citet{polin_observational_2019}. The model parameters are indicated in the legend as (C/O core mass $+$ He shell mass). The upper (lower) panel shows the evolution in $g$-band ($r$-band) absolute magnitudes. {\it Right:} Comparison of the spectrum of \sn\ with the $0.87\,\mathrm{M_\odot}$ C/O core $+$ $0.08\,\mathrm{M_\odot}$ He-shell DDet model before peak luminosity. Each spectrum is normalized by the median flux between 6500 and 7500\,\r{A}, and binned with a size of 10\,\r{A}. The synthetic spectrum 4\,days before the the $r$-band peak best matches the NOT spectrum (Galactic extinction corrected), which was obtained $\sim$4\,days before the $r_\mathrm{ZTF}$-band peak. All the phases have been rescaled to the host galaxy rest frame. }
\end{figure*}
\section{Discussion and Conclusions} \label{sec:discussion}
In the paper, we have presented the observations of \sn, a sub-luminous SN Ia. Putting together its unusual red colors and strong line-blanketing in the spectra near peak light, we show its peculiar nature as a SN Ia. It bears a high degree of resemblance to SN\,2018byg \citep{de_18byg_2019}, whose observational properties could be explained by the detonation of a shell of helium on a sub-\Mch WD. Fitting the light curve of \sn\ to a grid of models in \citet{polin_observational_2019}, we show a $\approx$0.87\,$\mathrm{M_\odot}$ WD beneath a $\approx$0.08\,$\mathrm{M_\odot}$ He-shell would be a reasonable estimate on its progenitor properties.

A high-SNR NIR spectrum obtained three weeks after the peak light shows a prominent absorption feature near 1\,\micron, which could be produced by the unburnt helium (\ion{He}{1} $\lambda$1.0830\,\micron) in the outermost ejecta expanding at a high velocity ($\approx$26,000\,km\,s$^{-1}$). At the same epoch, the \ion{Ca}{2} IRT also exhibits similarly high velocities ($\approx$24,000\,km\,s$^{-1}$). By now, we have a very small sample of four candidate He-shell DDet SNe which have NIR spectra observed. Interesting, all of them show deep absorption features near 1\,\micron, which, if assumed to have a helium origin, would be expanding very a similar velocity as the high velocity component of \ion{Ca}{2} IRT, despite the huge diversity in the \ion{Ca}{2} IRT velocity (from $\approx$15,000\,km\,s$^{-1}$ in SN\,2016dsg to $\approx$24,000\,km\,s$^{-1}$ in \sn) in our tiny sample. Such a consistency in velocities of absorption lines would be naturally explained if it is the unburnt helium and the newly synthesized calcium from the He-shell that produce these line features. However, we could not find unambiguous evidence for other \ion{He}{1} absorption lines, such as \ion{He}{1} $\lambda$2.0581\,\micron, so we are not drawing a strong conclusion of helium detection in \sn. Nonetheless, we discuss other potential strong lines (\ion{Mg}{2}, \ion{C}{1}, \ion{Fe}{2}) that may cause the 1\,\micron\ feature, but have found them also not that likely. Helium is still the most promising candidate.

This paper provides a framework of robust \ion{He}{1} detection in He-shell DDet SNe. Ideally, one will need a NIR spectrum covering both the 1\,\micron\ and 2\,\micron\ regions in search of the \ion{He}{1} $\lambda$1.0830\,\micron\ and $\lambda$2.0581\,\micron\ features. Since the \ion{He}{1} $\lambda$2.0581\,\micron\ is weaker or even invisible when the He-shell is thin \citep{Boyle2017_Helium}, and could be blended with strong telluric lines, one should not always expect to see significant absorption features near 2\,\micron. For those with a clear 1\,\micron\ feature, one can calculate the required velocity for an origin in \ion{He}{1} $\lambda$1.0830\,\micron, to see if it is comparable with the velocity with the HVFs in the \ion{Ca}{2} IRT at a similar phase. While the detonation recipe in a DDet model and the viewing angles would all affect the observed \ion{He}{1} velocity, we still expect the elements along the line-of-sight to expand at a similar pace, if they all have a He-shell origin. Excluding the possibility of other strong lines is also necessary, especially when the NIR spectrum is obtained before the peak of the SN, when there can be strong \ion{Mg}{2} and \ion{C}{1} absorption \citep{Hsiao_CSP_2019}.

In the tiny sample of DDet candidates to date, we have seen diversity in observational properties, including the peak luminosity, color evolution, chemical abundances and line velocities, which could be explained by a large variety of He-shell masses, WD masses, viewing angles, and the initial chemical compositions in the shell. In addition, they are discovered in both old and young stellar populations, \sn\ being the first unambiguous thick He-shell DDet candidate in a star-forming galaxy, suggesting their possible origins in a mixture of formation channels and progenitor systems.

\begin{acknowledgements}
\end{acknowledgements}

\facility{PO:1.2m (ZTF), PO:1.5m (SEDM), Hale (DBSP), NOT (ALFOSC), Shane (Kast Double spectrograph), Keck:I (LRIS), Keck:II (DEIMOS), Gemini:Gillett (GNIRS)}

\software{\texttt{astropy} \citep{Astropy_2013, Astropy_2018}, \texttt{emcee} \citep{emcee_2013}, \texttt{matplotlib} \citep{Matplotlib_2007}, \texttt{prospector} \citep{Johnson_prospector_2021}, \texttt{PypeIt} \citep{pypeit:zenodo}, \texttt{scikit-learn} \citep{scikit-learn}, \texttt{scipy} \citep{Scipy_2020}, \texttt{seaborn} \citep{Waskom_seaborn_2021}.}

\bibliography{SN2020jgb, software}
\bibliographystyle{aasjournal}

%% This command is needed to show the entire author+affiliation list when
%% the collaboration and author truncation commands are used.  It has to
%% go at the end of the manuscript.
%\allauthors

%% Include this line if you are using the \added, \replaced, \deleted
%% commands to see a summary list of all changes at the end of the article.
%\listofchanges

\end{document}

% End of file `sample631.tex'.
